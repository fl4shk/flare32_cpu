\documentclass{article}

\usepackage{graphicx}
\usepackage{float}
\usepackage{fancyvrb}
\usepackage[T1]{fontenc}
\usepackage{lmodern}
\usepackage{setspace}
\usepackage[nottoc]{tocbibind}
\usepackage[font=large]{caption}
\usepackage{framed}
\usepackage{tabularx}
\usepackage{amsmath}
%\usepackage{hyperref}
\usepackage[backend=biber,sorting=none]{biblatex}
%\usepackage[
%	backend=biber,
%	style=ieee,
%	sorting=none
%]{biblatex}
%\addbibresource{project_refs.bib}


\title{Flare32 CPU}
\date{2018-11-2}
\author{FL4SHK}

%% Hide section, subsection, and subsubsection numbering
%\renewcommand{\thesection}{}
%\renewcommand{\thesubsection}{}
%\renewcommand{\thesubsubsection}{}


% Alternative form of doing section stuff
\renewcommand{\thesection}{}
\renewcommand{\thesubsection}{\arabic{section}.\arabic{subsection}}
\makeatletter
\def\@seccntformat#1{\csname #1ignore\expandafter\endcsname\csname the#1\endcsname\quad}
\let\sectionignore\@gobbletwo
\let\latex@numberline\numberline
\def\numberline#1{\if\relax#1\relax\else\latex@numberline{#1}\fi}
\makeatother

\makeatletter
\renewcommand\tableofcontents{%
    \@starttoc{toc}%
}
\makeatother

%Figures
%\begin{figure}[H]
%	\includegraphics[width=\linewidth]{example.png}
%\end{figure}

% Verbatim text
%\VerbatimInput{main.cpp}

%% Fixed-width text
%\texttt{module FullAdder(input logic a, b, c_in, output logic s, c_out);}
%% Bold
%\textbf{green eggs}
%% Italic
%\textit{and}
%% Underline
%\underline{eggs}

%% Non-numbered list
%\begin{itemize}
%\item item 0
%\item item 1
%\end{itemize}

%% Numbered list
%\begin{enumerate}
%\item item 0
%\item item 1
%\end{enumerate}

%% Spaces and new lines
%LaTeX ignores extra spaces and new lines.
%Place \\ at the end of a line to create a new line (but not create a new
%paragraph)

%% Use "\noindent" to prevent a paragraph from indenting

%% Tables
%\begin{table}[H]
%	\begin{center}
%		\caption{Results for \texttt{blocksPerGrid = 5}}:
%		\label{tab:table0}
%		\begin{tabular}{c|c}
%			\textbf{\texttt{threadsPerBlock}}
%				& \textbf{\texttt{scaling()} Running Time (us)}\\
%			\hline
%			32 & 156.39\\
%			64 & 163.59\\
%			128 & 155.62\\
%			256 & 155.56\\
%			512 & 161.57\\
%			1024 & 166.85\\
%		\end{tabular}
%	\end{center}
%\end{table}

\begin{document}
	\maketitle
	\pagenumbering{gobble}
	\newpage
	\pagenumbering{arabic}


	\doublespacing
	\section{Abstract}
	\setcounter{section}{-1}
	%This is an instruction set designed to be similar to the SuperFX/GSU,
	%while not being binary compatible, nor even assembly language
	%compatible.  It takes some of the ideas from the GSU and runs with
	%them.
	This is an instruction set that is basically just my answer to THUMB
	(especially original THUMB), with some ideas taken from other places as
	well.  This shows how to get more out of 16-bit instructions than THUMB
	allows.

	\newpage
	\singlespacing
	\section{Table of Contents}
	\tableofcontents
	\newpage

	\doublespacing
	\section{Introduction}
	\subsection{Registers}
	There are sixteen general-purpose registers:  \texttt{r0}, \texttt{r1},
	\texttt{r2}, \texttt{...}, \texttt{r12}, \texttt{lr}, \texttt{fp},
	\texttt{sp}.  From the hardware's perspective, they are all equivalent. 
	Each register is 32 bits long.  For special purpose registers, there
	are also \texttt{pc}, the program counter (which is 32 bits long), and
	the \texttt{flags}.  Also there are the interrupts-related registers:
	\texttt{ids} (the destination to go to upon an interrupt happening),
	\texttt{ira} (the program counter value to return to after an
	interrupt) and \texttt{ie} (whether or not interrupts are enabled).
	Two more registers are \texttt{hi} and \texttt{lo}, which are used as
	the high 32 bits and low 32 bits of the result of a 32 by 32 -> 64
	multiplication, or as the high 32 bits and low 32 bits of the result of
	a 64 by 32 -> 64 division.
	Here are the flags:

	\begin{table}[H]
		\begin{center}
			\caption{The Flags}
			\label{tab:flags}
			\begin{tabular}{|c|c|c|c|}
				\hline
				Zero (Z) & Carry (C) & oVerflow (V) & Negative (N)\\
				\hline
			\end{tabular}
		\end{center}
	\end{table}

	\newpage
	\section{Instruction Set}
	While this machine takes inspiration from the GSU, it uses 16-bit
	instructions rather than 8-bit ones.

	\subsection{Instruction Group 0:  The \texttt{pre} Instruction}
	The following encoding is used, with each character representing one
	bit:  \\
	\texttt{000i iiii iiii iiii}, where 
	
	\singlespacing
	\begin{itemize}
		\item \texttt{i} is a 13-bit constant.
	\end{itemize}

	At the assembly language level, the instruction is simply
	\texttt{pre expr}, but it isn't typical to use this instruction
	directly in assembly language source code.  Instead, when an immediate
	value will not fit in a 5-bit signed immediate, this instruction will
	be inserted by the assembler right before the instruction using an
	immediate value.  The 13-bit constant will be used as bits
	\texttt{[17:5]} of the immediate value of the following instruction.
	The full 18-bit immediate value will then be sign extended to 32 bits.
	Also, if the previous instruction was a \texttt{pre}, interrupts will
	not be serviced.

	\subsection{Instruction Group 1:  ALU Instr with Immediate}
	The following encoding is used, with each character representing one
	bit:  \\
	\texttt{001i iiii aaaa ooof}, where

	\singlespacing
	\begin{itemize}
		\item \texttt{i} is a 5-bit sign-extended immediate
		\item \texttt{a} encodes register \texttt{rA}
		\item \texttt{o} is the opcode
		\item \texttt{f} is \texttt{0} if this instruction cannot affect
		flags and \texttt{1} if this instruction is permitted to affect
		flags. 
	\end{itemize}
	\doublespacing

	Here is a list of instructions from this encoding group.

	\singlespacing
	\begin{itemize}
		\item Opcode \texttt{0x0}:
			\texttt{\textbf{addi} rA, simm}
		\begin{itemize}
			\item Mnemonic for when flags not affected:  \texttt{addi}
			\item Mnemonic for when flags affected:  \texttt{addi.f}
			\item Affectable flags:
				\texttt{Z}, \texttt{C}, \texttt{V}, \texttt{N}
		\end{itemize}
		\item Opcode \texttt{0x1}:
			\texttt{\textbf{subi} rA, simm}
		\begin{itemize}
			\item Mnemonic for when flags not affected:  \texttt{subi}
			\item Mnemonic for when flags affected:  \texttt{subi.f}
			\item Affectable flags:
				\texttt{Z}, \texttt{C}, \texttt{V}, \texttt{N}
		\end{itemize}
		\item Opcode \texttt{0x2}:
			\texttt{\textbf{addi} rA, pc, simm}
		\begin{itemize}
			\item Mnemonic for when flags not affected:  \texttt{addi}
			\item Mnemonic for when flags affected:  \texttt{addi.f}
			\item Affectable flags:
				\texttt{Z}, \texttt{C}, \texttt{V}, \texttt{N}
		\end{itemize}
		\item Opcode \texttt{0x3}:
			\texttt{\textbf{cpyi} rA, simm}
		\begin{itemize}
			\item Mnemonic for when flags not affected:  \texttt{cpyi}
			\item Mnemonic for when flags affected:  \texttt{cpyi.f}
			\item Note:  Copy an immediate value into \texttt{rA}
			\item Affectable flags:
				\texttt{Z}, \texttt{N}
		\end{itemize}
		\item Opcode \texttt{0x4}:
			\texttt{\textbf{lui} rA, simm}
		\begin{itemize}
			\item Mnemonic for when flags not affected:  \texttt{lui}
			\item Mnemonic for when flags affected:  \texttt{lui.f}
			\item Note:  Copy bits \texttt{[15:0]} of the sign-extended
			immediate value into bits \texttt{[31:16]} of \texttt{rA}.
			Bits \texttt{[15:0]} of \texttt{rA} are not affected.
			\item Affectable flags:
				\texttt{Z}, \texttt{N}
		\end{itemize}
		\item Opcode \texttt{0x5}:
			\texttt{\textbf{lsli} rA, simm}
		\begin{itemize}
			\item Mnemonic for when flags not affected:  \texttt{lsli}
			\item Mnemonic for when flags affected:  \texttt{lsli.f}
			\item Note:  Logical shift left
			\item Affectable flags:
				\texttt{Z}, \texttt{N}
		\end{itemize}
		\item Opcode \texttt{0x6}:
			\texttt{\textbf{lsri} rA, simm}
		\begin{itemize}
			\item Mnemonic for when flags not affected:  \texttt{lsri}
			\item Mnemonic for when flags affected:  \texttt{lsri.f}
			\item Note:  Logical shift right
			\item Affectable flags:
				\texttt{Z}, \texttt{N}
		\end{itemize}
		\item Opcode \texttt{0x7}:
			\texttt{\textbf{asri} rA, simm}
		\begin{itemize}
			\item Mnemonic for when flags not affected:  \texttt{asri}
			\item Mnemonic for when flags affected:  \texttt{asri.f}
			\item Note:  Arithmetic shift right
			\item Affectable flags:
				\texttt{Z}, \texttt{N}
		\end{itemize}
	\end{itemize}

	\doublespacing

	\subsection{Instruction Group 2:  ALU Instr without Immediate}
	The following encoding is used, with each character representing one
	bit:  \\
	\texttt{010o cccc aaaa ooof}, where

	\singlespacing
	\begin{itemize}
		\item \texttt{o} is the opcode
		\item \texttt{c} encodes register \texttt{rC}
		\item \texttt{a} encodes register \texttt{rA}
		\item \texttt{f} is \texttt{0} if this instruction cannot affect
		flags and \texttt{1} if this instruction is permitted to affect
		flags. 
	\end{itemize}
	\doublespacing

	Here is a list of instructions from this encoding group.

	\singlespacing
	\begin{itemize}
		\item Opcode \texttt{0x0}:
			\texttt{\textbf{add} rA, rC}
		\begin{itemize}
			\item Mnemonic for when flags not affected:  \texttt{add}
			\item Mnemonic for when flags affected:  \texttt{add.f}
			\item Affectable flags:
				\texttt{Z}, \texttt{C}, \texttt{V}, \texttt{N}
		\end{itemize}
		\item Opcode \texttt{0x1}:
			\texttt{\textbf{sub} rA, rC}
		\begin{itemize}
			\item Mnemonic for when flags not affected:  \texttt{sub}
			\item Mnemonic for when flags affected:  \texttt{sub.f}
			\item Affectable flags:
				\texttt{Z}, \texttt{C}, \texttt{V}, \texttt{N}
		\end{itemize}
		\item Opcode \texttt{0x2}:
			\texttt{\textbf{add} rA, pc, rC}
		\begin{itemize}
			\item Mnemonic for when flags not affected:  \texttt{add}
			\item Mnemonic for when flags affected:  \texttt{add.f}
			\item Affectable flags:
				\texttt{Z}, \texttt{C}, \texttt{V}, \texttt{N}
		\end{itemize}
		\item Opcode \texttt{0x3}:
			\texttt{\textbf{cpy} rA, rC}
		\begin{itemize}
			\item Mnemonic for when flags not affected:  \texttt{cpy}
			\item Mnemonic for when flags affected:  \texttt{cpy.f}
			\item Note:  Copy \texttt{rC} into \texttt{rA}
			\item Affectable flags:
				\texttt{Z}, \texttt{N}
		\end{itemize}
		\item Opcode \texttt{0x4}:
			\texttt{\textbf{lur} rA, rC}
		\begin{itemize}
			\item Mnemonic for when flags not affected:  \texttt{lur}
			\item Mnemonic for when flags affected:  \texttt{lur.f}
			\item Note:  Copy \texttt{b[15:0]} to \texttt{a[31:16]}.
			Bits \texttt{[15:0]} of \texttt{rA} are not affected.
			\item Affectable flags:
				\texttt{Z}, \texttt{N}
		\end{itemize}
		\item Opcode \texttt{0x5}:
			\texttt{\textbf{lsl} rA, rC}
		\begin{itemize}
			\item Mnemonic for when flags not affected:  \texttt{lsl}
			\item Mnemonic for when flags affected:  \texttt{lsl.f}
			\item Note:  Logical shift left
			\item Affectable flags:
				\texttt{Z}, \texttt{N}
		\end{itemize}
		\item Opcode \texttt{0x6}:
			\texttt{\textbf{lsr} rA, rC}
		\begin{itemize}
			\item Mnemonic for when flags not affected:  \texttt{lsr}
			\item Mnemonic for when flags affected:  \texttt{lsr.f}
			\item Note:  Logical shift right
			\item Affectable flags:
				\texttt{Z}, \texttt{N}
		\end{itemize}
		\item Opcode \texttt{0x7}:
			\texttt{\textbf{asr} rA, rC}
		\begin{itemize}
			\item Mnemonic for when flags not affected:  \texttt{asr}
			\item Mnemonic for when flags affected:  \texttt{asr.f}
			\item Note:  Arithmetic shift right
			\item Affectable flags:
				\texttt{Z}, \texttt{N}
		\end{itemize}
		\item Opcode \texttt{0x8}:
			\texttt{\textbf{adc} rA, rC}
		\begin{itemize}
			\item Mnemonic for when flags not affected:  \texttt{adc}
			\item Mnemonic for when flags affected:  \texttt{adc.f}
			\item Note:  Add with Carry
			\item Affectable flags:
				\texttt{Z}, \texttt{C}, \texttt{V}, \texttt{N}
		\end{itemize}
		\item Opcode \texttt{0x9}:
			\texttt{\textbf{sbc} rA, rC}
		\begin{itemize}
			\item Mnemonic for when flags not affected:  \texttt{sbc}
			\item Mnemonic for when flags affected:  \texttt{sbc.f}
			\item Note:  Subtract with Borrow
			\item Affectable flags:
				\texttt{Z}, \texttt{C}, \texttt{V}, \texttt{N}
		\end{itemize}
		\item Opcode \texttt{0xa}:
			\texttt{\textbf{cmp} rA, rC}
		\begin{itemize}
			\item Note:  Compare \texttt{rA} to \texttt{rC}.  \texttt{cmp}
			is \textit{always} able to affect flags, independent of the
			encoded \texttt{f} bit of the instruction.
			\item Affectable flags:
				\texttt{Z}, \texttt{C}, \texttt{V}, \texttt{N}
		\end{itemize}
		\item Opcode \texttt{0xb}:
			\texttt{\textbf{and} rA, rC}
		\begin{itemize}
			\item Mnemonic for when flags not affected:  \texttt{and}
			\item Mnemonic for when flags affected:  \texttt{and.f}
			\item Note:  Bitwise AND
			\item Affectable flags:
				\texttt{Z}, \texttt{N}
		\end{itemize}
		\item Opcode \texttt{0xc}:
			\texttt{\textbf{orr} rA, rC}
		\begin{itemize}
			\item Mnemonic for when flags not affected:  \texttt{orr}
			\item Mnemonic for when flags affected:  \texttt{orr.f}
			\item Note:  Bitwise OR
			\item Affectable flags:
				\texttt{Z}, \texttt{N}
		\end{itemize}
		\item Opcode \texttt{0xd}:
			\texttt{\textbf{xor} rA, rC}
		\begin{itemize}
			\item Mnemonic for when flags not affected:  \texttt{xor}
			\item Mnemonic for when flags affected:  \texttt{xor.f}
			\item Note:  Bitwise XOR
			\item Affectable flags:
				\texttt{Z}, \texttt{N}
		\end{itemize}
		\item Opcode \texttt{0xe}:
			\texttt{\textbf{seb} rA, rC}
		\begin{itemize}
			\item Mnemonic when flags not affected:  \texttt{seb}
			\item Mnemonic when flags affected:  \texttt{seb.f}
			\item Effect:
				Sign-extend \texttt{rC[7:0]} to 32 bits, then copy that
				value to \texttt{rA}
			\item Affectable flags:
				\texttt{Z}, \texttt{N}
		\end{itemize}
		\item Opcode \texttt{0xf}:
			\texttt{\textbf{seh} rA, rC}
		\begin{itemize}
			\item Mnemonic when flags not affected:  \texttt{seh}
			\item Mnemonic when flags affected:  \texttt{seh.f}
			\item Effect:
				Sign-extend \texttt{rC[15:0]} to 32 bits, then copy that
				value to \texttt{rA}
			\item Affectable flags:
				\texttt{Z}, \texttt{N}
		\end{itemize}
		%\item Opcode \texttt{0xe}:
		%	\texttt{\textbf{mul} rA, rC}
		%\begin{itemize}
		%	\item Mnemonic for when flags not affected:  \texttt{mul}
		%	\item Mnemonic for when flags affected:  \texttt{mul.f}
		%	\item Affectable flags:
		%		\texttt{Z}, \texttt{N}
		%\end{itemize}
		%\item Opcode \texttt{0xe}:
		%	\texttt{\textbf{ludiv} rA}
		%\begin{itemize}
		%	\item Mnemonic for when flags not affected:  \texttt{ludiv}
		%	\item Mnemonic for when flags affected:  \texttt{ludiv.f}
		%	\item Note:  performs a 64-bit by 32-bit unsigned division of
		%	\texttt{\{hi, lo\}} by \texttt{rA}, storing result in
		%	\texttt{\{hi, lo\}}.
		%	\item Affectable flags:
		%		\texttt{Z}, \texttt{N}
		%\end{itemize}
		%\item Opcode \texttt{0xf}:
		%	\texttt{\textbf{lsdiv} rA}
		%\begin{itemize}
		%	\item Mnemonic for when flags not affected:  \texttt{lsdiv}
		%	\item Mnemonic for when flags affected:  \texttt{lsdiv.f}
		%	\item Note:  performs a 64-bit by 32-bit signed division of
		%	\texttt{\{hi, lo\}} by \texttt{rA}, storing result in
		%	\texttt{\{hi, lo\}}.
		%	\item Affectable flags:
		%		\texttt{Z}, \texttt{N}
		%\end{itemize}
	\end{itemize}

	\doublespacing
	\subsection{Instruction Group 3:  Jumps and Stack Stuff}
	The following encoding is used, with each character representing one
	bit:  \\
	\texttt{011o ooo0 0000 bbbb}, where

	\singlespacing
	\begin{itemize}
		\item \texttt{o} is the opcode
		\item \texttt{b} encodes register \texttt{rB}
	\end{itemize}

	\doublespacing

	%At the assembly language level, \texttt{pick} is simply
	%\texttt{pick rB}, but as with \texttt{pre}, it isn't typical to use
	%this instruction directly in assembly language source code.  Instead,
	%when an instruction is typed as \texttt{instr rA, rB, operand}
	%\textit{and} when \texttt{rA} is not the same register as \texttt{rB},
	%a \texttt{pick rB} instruction will be inserted by the assembler before
	%the instruction in the source code.  As mentioned before, when the
	%instruction previous to the current one is either a \texttt{pre} or a
	%\texttt{pick}, interrupts will not be serviced.  Note that since
	%\texttt{pre}'s 13-bit partial immediate only affects the instruction
	%following the \texttt{pre}, only a 5-bit signed immediate can be used
	%with \texttt{pick} and an instruction from group 1.  For all
	%instructions, the default value of \texttt{rB} is \texttt{rA}, except
	%when the previous instruction was \texttt{pick}.

	Here is the list of instructions from this encoding group.

	\singlespacing
	\begin{itemize}
		\item Opcode \texttt{0x0}:
			\texttt{\textbf{jnv} rB}
		\begin{itemize}
			\item Effect:
				\texttt{if (0) pc <= rB;} This is a NOP.
		\end{itemize}
		\item Opcode \texttt{0x1}:
			\texttt{\textbf{jmp} rB}
		\begin{itemize}
			\item Effect:
				\texttt{if (1) pc <= rB;}
		\end{itemize}
		\item Opcode \texttt{0x2}:
			\texttt{\textbf{jge} rB}
		\begin{itemize}
			\item Effect:
				\texttt{if (!(flags.N XOR flags.V)) pc <= rB;}
		\end{itemize}
		\item Opcode \texttt{0x3}:
			\texttt{\textbf{jlt} rB}
		\begin{itemize}
			\item Effect:
				\texttt{if (flags.N XOR flags.V) pc <= rB;}
		\end{itemize}
		\item Opcode \texttt{0x4}:
			\texttt{\textbf{jne} rB}
		\begin{itemize}
			\item Effect:
				\texttt{if (!flags.Z) pc <= rB;}
		\end{itemize}
		\item Opcode \texttt{0x5}:
			\texttt{\textbf{jeq} rB}
		\begin{itemize}
			\item Effect:
				\texttt{if (flags.Z) pc <= rB;}
		\end{itemize}
		\item Opcode \texttt{0x6}:
			\texttt{\textbf{jpl} rB}
		\begin{itemize}
			\item Effect:
				\texttt{if (!flags.N) pc <= rB;}
		\end{itemize}
		\item Opcode \texttt{0x7}:
			\texttt{\textbf{jmi} rB}
		\begin{itemize}
			\item Effect:
				\texttt{if (flags.N) pc <= rB;}
		\end{itemize}
		\item Opcode \texttt{0x8}:
			\texttt{\textbf{jcc} rB}
		\begin{itemize}
			\item Effect:
				\texttt{if (!flags.C) pc <= rB;}
		\end{itemize}
		\item Opcode \texttt{0x9}:
			\texttt{\textbf{jcs} rB}
		\begin{itemize}
			\item Effect:
				\texttt{if (flags.C) pc <= rB;}
		\end{itemize}
		\item Opcode \texttt{0xa}:
			\texttt{\textbf{jvc} rB}
		\begin{itemize}
			\item Effect:
				\texttt{if (!flags.V) pc <= rB;}
		\end{itemize}
		\item Opcode \texttt{0xb}:
			\texttt{\textbf{jvs} rB}
		\begin{itemize}
			\item Effect:
				\texttt{if (flags.V) pc <= rB;}
		\end{itemize}
		%\item Opcode \texttt{0xc}:
		%	\texttt{\textbf{pick} rB}
	\end{itemize}

	\doublespacing

	\subsection{Instruction Group 4:  Relative Branches}
	The following encoding is used, with each character representing one
	bit:  \\
	\texttt{100o oooi iiii iiii}, where

	\singlespacing
	\begin{itemize}
		\item \texttt{o} is the opcode
		\item \texttt{i} is a 9-bit sign-extended immediate, which can
		still be affected by \texttt{pre}, but when \texttt{pre} is the
		instruction before this one, only bits \texttt{[4:0]} of the
		immediate value encoded into this instruction will be used by the
		hardware, with the remaining bits of the immediate coming from
		\texttt{pre}.  Thus, \texttt{pre} can still only form a 17-bit
		immediate for relative branches.
	\end{itemize}
	\doublespacing

	Here is a list of instructions from this encoding group.

	\singlespacing
	\begin{itemize}
		\item Opcode \texttt{0x0}:
			\texttt{\textbf{bnv} simm}
		\begin{itemize}
			\item Effect:
				\texttt{if (0) pc <= pc + simm + 2;} This is a NOP.
		\end{itemize}
		\item Opcode \texttt{0x1}:
			\texttt{\textbf{bra} simm}
		\begin{itemize}
			\item Effect:
				\texttt{if (1) pc <= pc + simm + 2;}
		\end{itemize}
		\item Opcode \texttt{0x2}:
			\texttt{\textbf{bge} simm}
		\begin{itemize}
			\item Effect:
				\texttt{if (!(flags.N XOR flags.V)) pc <= pc + simm + 2;}
		\end{itemize}
		\item Opcode \texttt{0x3}:
			\texttt{\textbf{blt} simm}
		\begin{itemize}
			\item Effect:
				\texttt{if (flags.N XOR flags.V) pc <= pc + simm + 2;}
		\end{itemize}
		\item Opcode \texttt{0x4}:
			\texttt{\textbf{bne} simm}
		\begin{itemize}
			\item Effect:
				\texttt{if (!flags.Z) pc <= pc + simm + 2;}
		\end{itemize}
		\item Opcode \texttt{0x5}:
			\texttt{\textbf{beq} simm}
		\begin{itemize}
			\item Effect:
				\texttt{if (flags.Z) pc <= pc + simm + 2;}
		\end{itemize}
		\item Opcode \texttt{0x6}:
			\texttt{\textbf{bpl} simm}
		\begin{itemize}
			\item Effect:
				\texttt{if (!flags.N) pc <= pc + simm + 2;}
		\end{itemize}
		\item Opcode \texttt{0x7}:
			\texttt{\textbf{bmi} simm}
		\begin{itemize}
			\item Effect:
				\texttt{if (flags.N) pc <= pc + simm + 2;}
		\end{itemize}
		\item Opcode \texttt{0x8}:
			\texttt{\textbf{bcc} simm}
		\begin{itemize}
			\item Effect:
				\texttt{if (!flags.C) pc <= pc + simm + 2;}
		\end{itemize}
		\item Opcode \texttt{0x9}:
			\texttt{\textbf{bcs} simm}
		\begin{itemize}
			\item Effect:
				\texttt{if (flags.C) pc <= pc + simm + 2;}
		\end{itemize}
		\item Opcode \texttt{0xa}:
			\texttt{\textbf{bvc} simm}
		\begin{itemize}
			\item Effect:
				\texttt{if (!flags.V) pc <= pc + simm + 2;}
		\end{itemize}
		\item Opcode \texttt{0xb}:
			\texttt{\textbf{bvs} simm}
		\begin{itemize}
			\item Effect:
				\texttt{if (flags.V) pc <= pc + simm + 2;}
		\end{itemize}
	\end{itemize}

	\doublespacing
	\subsection{Instruction Group 5:  Indexed Load}
	The following encoding is used, with each character representing one
	bit:  \\
	\texttt{101i cccc aaaa iiii}, where

	\singlespacing
	\begin{itemize}
		\item \texttt{i} is a 5-bit sign-extended immediate, which can
		be expanded by \texttt{pre}  
		\item \texttt{c} encodes register \texttt{rC}
		\item \texttt{a} encodes register \texttt{rA}
	\end{itemize}
	\doublespacing

	The one instruction from this encoding group is
	\texttt{\textbf{ldr} rA, [rC, simm]}.
	This is a 32-bit load into \texttt{rA}, where the effective address to
	load from is computed as \texttt{rC + simm}, using the sign-extended
	form of \texttt{simm}.
	%Note that \texttt{pick} does not affect this instruction in any way.

	\subsection{Instruction Group 6:  Indexed Store}
	The following encoding is used, with each character representing one
	bit:  \\
	\texttt{110i cccc aaaa iiii}, where

	\singlespacing
	\begin{itemize}
		\item \texttt{i} is a 5-bit sign-extended immediate, which can
		be expanded by \texttt{pre}  
		\item \texttt{c} encodes register \texttt{rC}
		\item \texttt{a} encodes register \texttt{rA}
	\end{itemize}
	\doublespacing

	The one instruction from this encoding group is
	\texttt{\textbf{str} rA, [rC, simm]}.
	This is a 32-bit store of \texttt{rA}, where the effective address to
	store to is computed as \texttt{rC + simm}, using the sign-extended
	form of \texttt{simm}.
	%Note that \texttt{pick} does not affect this instruction in any way.

	\subsection{Instruction Group 7.0:  Other Loads and Stores}
	The following encoding is used, with each character representing one
	bit:  \\
	\texttt{1110 cccc aaaa 00oo}, where

	\singlespacing
	\begin{itemize}
		\item \texttt{c} encodes register \texttt{rC}
		\item \texttt{a} encodes register \texttt{rA}
		\item \texttt{o} is the opcode
	\end{itemize}
	\doublespacing

	Here is a list of instructions from this encoding group.

	\singlespacing
	\begin{itemize}
		\item Opcode \texttt{0x0}:
			\texttt{\textbf{ldb} rA, [rC]}
		\begin{itemize}
			\item Effect:  Load a byte from memory at address computed as
			just the value of \texttt{rC}, zero-extend the byte to 32 bits,
			then put the zero-extended 32-bit value into \texttt{rA}.
		\end{itemize}
		\item Opcode \texttt{0x1}:
			\texttt{\textbf{ldh} rA, [rC]}
		\begin{itemize}
			\item Effect:  Load a 16-bit value from memory at address
			computed as just the value of \texttt{rC}, zero-extend the
			16-bit value to 32 bits, then put the zero-extended 32-bit
			value into \texttt{rA}.
		\end{itemize}
		\item Opcode \texttt{0x2}:
			\texttt{\textbf{stb} rA, [rC]}
		\begin{itemize}
			\item Effect:  Store \texttt{rA[7:0]} to memory at the address
			computed as just the value of \texttt{rC}.
		\end{itemize}
		\item Opcode \texttt{0x3}:
			\texttt{\textbf{sth} rA, [rC]}
		\begin{itemize}
			\item Effect:  Store \texttt{rA[15:0]} to memory at the address
			computed as just the value of \texttt{rC}.
		\end{itemize}
	\end{itemize}
	\doublespacing

	\subsection{Instruction Group 7.1:  Copying to/from Special Registers}
	The following encoding is used, with each character representing one
	bit:  \\
	\texttt{1110 oooo aaaa 0100}, where

	\singlespacing
	\begin{itemize}
		\item \texttt{o} is the opcode
		\item \texttt{a} encodes register \texttt{rA}
	\end{itemize}
	\doublespacing

	Here is a list of instructions from this encoding group.

	\singlespacing
	\begin{itemize}
		\item Opcode \texttt{0x0}:
			\texttt{\textbf{cpyspec} hi, rA}
		\item Opcode \texttt{0x1}:
			\texttt{\textbf{cpyspec} rA, hi}
		\item Opcode \texttt{0x2}:
			\texttt{\textbf{cpyspec} lo, rA}
		\item Opcode \texttt{0x3}:
			\texttt{\textbf{cpyspec} rA, lo}
		\item Opcode \texttt{0x4}:
			\texttt{\textbf{cpyspec} flags, rA}
		\begin{itemize}
			\item Effect:  copy \texttt{rA[3:0]} to \texttt{flags}.
		\end{itemize}
		\item Opcode \texttt{0x5}:
			\texttt{\textbf{cpyspec} rA, flags}
		\begin{itemize}
			\item Effect:  copy \texttt{flags} to \texttt{rA}.
		\end{itemize}
		\item Opcode \texttt{0x6}:
			\texttt{\textbf{cpyspec} ids, rA}
		\item Opcode \texttt{0x7}:
			\texttt{\textbf{cpyspec} rA, ids}
		\item Opcode \texttt{0x8}:
			\texttt{\textbf{cpyspec} ira, rA}
		\item Opcode \texttt{0x9}:
			\texttt{\textbf{cpyspec} rA, ira}
		\item Opcode \texttt{0xa}:
			\texttt{\textbf{cpyspec} ie, rA}
		\begin{itemize}
			\item Effect:  copy \texttt{rA[0]} to \texttt{ie}.
		\end{itemize}
		\item Opcode \texttt{0xb}:
			\texttt{\textbf{cpyspec} rA, ie}
		\begin{itemize}
			\item Effect:  copy \texttt{ie} to \texttt{rA}.
		\end{itemize}
		\item Opcode \texttt{0xc}:
			\texttt{\textbf{ei}}
		\begin{itemize}
			\item Effect:  copy \texttt{1} into \texttt{ie}.
		\end{itemize}
		\item Opcode \texttt{0xd}:
			\texttt{\textbf{di}}
		\begin{itemize}
			\item Effect:  copy \texttt{0} into \texttt{ie}.
		\end{itemize}
		\item Opcode \texttt{0xe}:
			\texttt{\textbf{push} ira, rA}
		\begin{itemize}
			\item Effect:  pushes \texttt{ira} onto the stack, using
			register \texttt{rA} as the stack pointer register.  It is
			expected that the typical \texttt{rA} will simply be
			\texttt{sp}.
		\end{itemize}
		\item Opcode \texttt{0xf}:
			\texttt{\textbf{pop} ira, rA}
		\begin{itemize}
			\item Effect:  pops \texttt{ira} off the stack, using
			register \texttt{rA} as the stack pointer register.  It is
			expected that the typical \texttt{rA} will simply be
			\texttt{sp}.
		\end{itemize}
	\end{itemize}
	\doublespacing

	\subsection{Instruction Group 7.2:  Multiplies, Divides, and
	Extra Interrupts Stuff}
	The following encoding is used, with each character representing one
	bit:  \\
	\texttt{1110 oooo aaaa 1000}, where

	\singlespacing
	\begin{itemize}
		\item \texttt{o} is the opcode
		\item \texttt{a} encodes register \texttt{rA}
	\end{itemize}
	\doublespacing

	Hereis a list of instructions from this encoding group.

	\singlespacing
	\begin{itemize}
		\item Opcode \texttt{0x0}:
			\texttt{\textbf{lumul}}
		\begin{itemize}
			\item Effect:  This instruction multiplies \texttt{hi} by
			\texttt{lo}, performing an unsigned 32-bit by 32-bit -> 64-bit
			multiply, storing result in \texttt{\{hi, lo\}}.
		\end{itemize}
		\item Opcode \texttt{0x1}:
			\texttt{\textbf{lsmul}}
		\begin{itemize}
			\item Effect:  This instruction multiplies \texttt{hi} by
			\texttt{lo}, performing a signed 32-bit by 32-bit -> 64-bit
			multiply, storing result in \texttt{\{hi, lo\}}.
		\end{itemize}
		\item Opcode \texttt{0x2}:
			\texttt{\textbf{ludiv} rA}
		\begin{itemize}
			\item Effect:  performs a 64-bit by 32-bit unsigned division of
			\texttt{\{hi, lo\}} by \texttt{rA}, storing 64-bit result in
			\texttt{\{hi, lo\}}.
		\end{itemize}
		\item Opcode \texttt{0x3}:
			\texttt{\textbf{lsdiv} rA}
		\begin{itemize}
			\item Effect:  performs a 64-bit by 32-bit signed division of
			\texttt{\{hi, lo\}} by \texttt{rA}, storing 64-bit result in
			\texttt{\{hi, lo\}}.
		\end{itemize}
		\item Opcode \texttt{0x4}:
			\texttt{\textbf{reti}}
		\begin{itemize}
			\item Effect:  enables interrupts and performs
			\texttt{pc <= ira;}
		\end{itemize}
		\item Opcode \texttt{0x5}:
			\texttt{\textbf{jmp} ira}
		\begin{itemize}
			\item Effect:  \texttt{pc <= ira;}
		\end{itemize}
	\end{itemize}
	\doublespacing

	%\printbibliography[heading=bibnumbered,title={Bibliography}]

\end{document}

