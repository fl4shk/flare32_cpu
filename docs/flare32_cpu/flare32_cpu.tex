\documentclass{article}

\usepackage{graphicx}
\usepackage{float}
\usepackage{fancyvrb}
\usepackage[T1]{fontenc}
\usepackage{lmodern}
\usepackage{setspace}
\usepackage[nottoc]{tocbibind}
\usepackage[font=large]{caption}
\usepackage{framed}
\usepackage{tabularx}
\usepackage{amsmath}
%\usepackage{hyperref}
\usepackage[backend=biber,sorting=none]{biblatex}
%\usepackage[
%	backend=biber,
%	style=ieee,
%	sorting=none
%]{biblatex}
%\addbibresource{project_refs.bib}


\title{Flare32 CPU}
%\date{2018-11-2}
\author{FL4SHK}

%% Hide section, subsection, and subsubsection numbering
%\renewcommand{\thesection}{}
%\renewcommand{\thesubsection}{}
%\renewcommand{\thesubsubsection}{}


% Alternative form of doing section stuff
\renewcommand{\thesection}{}
\renewcommand{\thesubsection}{\arabic{section}.\arabic{subsection}}
\makeatletter
\def\@seccntformat#1{\csname #1ignore\expandafter\endcsname\csname the#1\endcsname\quad}
\let\sectionignore\@gobbletwo
\let\latex@numberline\numberline
\def\numberline#1{\if\relax#1\relax\else\latex@numberline{#1}\fi}
\makeatother

\makeatletter
\renewcommand\tableofcontents{%
    \@starttoc{toc}%
}
\makeatother

%Figures
%\begin{figure}[H]
%	\includegraphics[width=\linewidth]{example.png}
%\end{figure}

% Verbatim text
%\VerbatimInput{main.cpp}

%% Fixed-width text
%\texttt{module FullAdder(input logic a, b, c_in, output logic s, c_out);}
%% Bold
%\textbf{green eggs}
%% Italic
%\textit{and}
%% Underline
%\underline{eggs}

%% Non-numbered list
%\begin{itemize}
%\item item 0
%\item item 1
%\end{itemize}

%% Numbered list
%\begin{enumerate}
%\item item 0
%\item item 1
%\end{enumerate}

%% Spaces and new lines
%LaTeX ignores extra spaces and new lines.
%Place \\ at the end of a line to create a new line (but not create a new
%paragraph)

%% Use "\noindent" to prevent a paragraph from indenting

%% Tables
%\begin{table}[H]
%	\begin{center}
%		\caption{Results for \texttt{blocksPerGrid = 5}}:
%		\label{tab:table0}
%		\begin{tabular}{c|c}
%			\textbf{\texttt{threadsPerBlock}}
%				& \textbf{\texttt{scaling()} Running Time (us)}\\
%			\hline
%			32 & 156.39\\
%			64 & 163.59\\
%			128 & 155.62\\
%			256 & 155.56\\
%			512 & 161.57\\
%			1024 & 166.85\\
%		\end{tabular}
%	\end{center}
%\end{table}

\begin{document}
	\maketitle
	\pagenumbering{gobble}
	\newpage
	\pagenumbering{arabic}


	%\doublespacing
	%\section{Abstract}
	%\setcounter{section}{-1}
	%%This is an instruction set designed to be similar to the SuperFX/GSU,
	%%while not being binary compatible, nor even assembly language
	%%compatible.  It takes some of the ideas from the GSU and runs with
	%%them.
	%This is an instruction set that is basically just my answer to THUMB
	%(especially original THUMB), with some ideas taken from other places as
	%well.  This shows how to get more out of 16-bit instructions than THUMB
	%allows.

	\newpage
	\singlespacing
	\section{Table of Contents}
	\tableofcontents
	\newpage

	\doublespacing
	\section{Introduction}
	\subsection{Registers}
	There are sixteen general-purpose registers:  \texttt{r0}, \texttt{r1},
	\texttt{r2}, \texttt{...}, \texttt{r11}, \texttt{r12}, \texttt{lr},
	\texttt{fp}, \texttt{sp}.  Each register
	is 32 bits long.  For special purpose registers, there are also
	\texttt{pc}, the program counter (which is 32 bits long), and the
	\texttt{flags}.  Also there are the interrupts-related registers:
	\texttt{ids} (the destination to go to upon an interrupt happening),
	\texttt{ira} (the program counter value to return to after an
	interrupt) and \texttt{ie} (whether or not interrupts are enabled).
	Two more registers are \texttt{hi} and \texttt{lo}, which are used as
	the high 32 bits and low 32 bits of the result of a 32 by 32 -> 64
	multiplication, or as the high 32 bits and low 32 bits of the result of
	a 64 by 32 -> 64 division.  Here are the flags:

	\begin{table}[H]
		\begin{center}
			\caption{The Flags}
			\label{tab:flags}
			\begin{tabular}{|c|c|c|c|}
				\hline
				Zero (Z) & Carry (C) & oVerflow (V) & Negative (N)\\
				\hline
			\end{tabular}
		\end{center}
	\end{table}

	\newpage
	\section{Instruction Set}
	%While this machine takes inspiration from the GSU, it uses 16-bit
	%instructions rather than 8-bit ones.

	\subsection{Instruction Group 0:  The \texttt{pre} Instruction}
	The following encoding is used, with each character representing one
	bit:  \\
	\texttt{000i iiii iiii iiii}, where 
	
	\singlespacing
	\begin{itemize}
		\item \texttt{i} is a 13-bit constant.
	\end{itemize}

	%At the assembly language level, the instruction is simply
	%\texttt{pre expr}, but it isn't typical to use this instruction
	%directly in assembly language source code.  Instead, when an immediate
	%value will not fit in a 5-bit signed immediate, this instruction will
	%be inserted by the assembler right before the instruction using an
	%immediate value.  The 13-bit constant will be used as bits
	%\texttt{[17:5]} of the immediate value of the following instruction.
	%The full 18-bit immediate value will then be sign extended to 32 bits.
	%Also, if the previous instruction was a \texttt{pre}, interrupts will
	%not be serviced.

	%Multiple \texttt{pre} instructions can be used to form even larger
	%immediates.  If two \texttt{pre} instructions were used, for example, a
	%31-bit immediate can be used.  While not needed in most cases, three
	%\texttt{pre} instructions permit use of 32-bit immediates.
	\texttt{pre} is a mechanism by which immediates larger than 5-bit can
	be used, kind of acting like variable width instructions.
	
	If one \texttt{pre} instruction is used, the immediate field of the
	\texttt{pre} instruction will form bits \texttt{[17:5]} of the
	immediate of the following instruction.  The 18-bit immediate will then
	be sign-extended to 32 bits.

	If two \texttt{pre} instructions are used, the first \texttt{pre}
	instruction forms bits \texttt{[17:5]} of the immediate, and the second
	\texttt{pre} instruction forms bits \texttt{[30:18]} of the immediate.
	The full 31-bit immediate will be sign-extended to 32 bits.

	If three \texttt{pre} instructions are used, the first and second
	\texttt{pre} instructions are treated like when there are only two
	\texttt{pre} instructions, and the final/third \texttt{pre} instruction
	forms just bit \texttt{[31]} of the 32-bit immediate.  In this case, no
	sign extension will be performed.

	It is expected that the assembler (or linker) will be the one to insert
	\texttt{pre} instructions as needed.  When a \texttt{pre} instruction
	happens, \texttt{pre} is considered to be "in effect" and will stick
	around until certain conditions are met.  Also, if \texttt{pre} is in
	effect, interrupts will not be serviced.

	Any time \texttt{pre} stops being in effect, \texttt{index} will stop
	being in effect as well.  \texttt{index} is defined later.

	%If \texttt{pre} goes into effect, then the first non-\texttt{pre},
	%non-\texttt{index} instruction that is executed while \texttt{pre} is
	%in effect will store that \texttt{pre} is no longer in effect. 

	%If a \texttt{pre} instruction is found when there are three
	%\texttt{pre} instructions in effect, then the instruction will be
	%treated as a NOP, and both \texttt{pre} and \texttt{index} will stop
	%being in effect.

	%Otherwise, if \texttt{pre} is in effect, then the first
	%non-\texttt{pre}, non-\texttt{index} instruction that is executed while
	%\texttt{pre} is in effect will store that \texttt{pre} is no longer in
	%effect.

	If three \texttt{pre} instructions are in effect and a fourth one is
	found, it will be treated as a NOP, and \texttt{pre} will stop being in
	effect. 

	If at least one \texttt{pre} is in effect and there is also an
	\texttt{index} in effect, then if the current instruction is
	\texttt{index}, the previous \texttt{pre}(s) and \texttt{index} will
	stop being in effect, and the instruction (which is a second
	\texttt{index}) will be treated as a NOP.

	In other cases, once a non-\texttt{pre}, non-\texttt{index} instruction
	is found, \texttt{pre} will stop being in effect, regardless of whether
	or not its value was used for the instruction.  The same is true if
	\texttt{index} was in effect.

	%If four \texttt{pre} instructions are used, the (effectively 64-bit)
	%instruction will be treated as a \texttt{NOP}.

	%In any of these cases, if \texttt{pre} was the previous instruction,
	%the current instruction cannot be interrupted.


	\subsection{Instruction Group 1}
	The following encoding is used, with each character representing one
	bit:  \\
	\texttt{001i iiii oooo aaaa}, where

	\singlespacing
	\begin{itemize}
		\item \texttt{i} is a 5-bit sign-extended immediate, and is denoted
		\texttt{simm}
		\item \texttt{a} encodes register \texttt{rA}
		\item \texttt{o} is the opcode
		%\item \texttt{f} is \texttt{0} if this instruction cannot affect
		%flags and \texttt{1} if this instruction is permitted to affect
		%flags. 
	\end{itemize}
	\doublespacing

	Here is a list of instructions from this encoding group.

	\singlespacing
	\begin{itemize}
		\item Opcode \texttt{0x0}:
			\texttt{\textbf{addi} rA, \#simm}
		\item Opcode \texttt{0x1}:
			% subi was deemed useless given that immediates are
			% sign-extended.
			\texttt{\textbf{addi} rA, pc, \#simm}
		\item Opcode \texttt{0x2}
			\texttt{\textbf{addi} rA, sp, \#simm}
		\item Opcode \texttt{0x3}:
			\texttt{\textbf{addi} rA, fp, \#simm}
		\item Opcode \texttt{0x4}:
			\texttt{\textbf{cmpi} rA, \#simm}
		\begin{itemize}
			\item Note:  Compare \texttt{rA} to \texttt{simm}.
			\item Affectable flags:
				\texttt{Z}, \texttt{C}, \texttt{V}, \texttt{N}
		\end{itemize}
		\item Opcode \texttt{0x5}:
			\texttt{\textbf{cpyi} rA, \#simm}
		\begin{itemize}
			\item Note:  Copy an immediate value into \texttt{rA}
		\end{itemize}
		\item Opcode \texttt{0x6}:
			\texttt{\textbf{lsli} rA, \#simm}
		\begin{itemize}
			\item Note:  Logical shift left
		\end{itemize}
		\item Opcode \texttt{0x7}:
			\texttt{\textbf{lsri} rA, \#simm}
		\begin{itemize}
			\item Note:  Logical shift right
		\end{itemize}
		\item Opcode \texttt{0x8}:
			\texttt{\textbf{asri} rA, \#simm}
		\begin{itemize}
			\item Note:  Arithmetic shift right
		\end{itemize}
		\item Opcode \texttt{0x9}:
			\texttt{\textbf{andi} rA, \#simm}
		\begin{itemize}
			\item Note:  Bitwise AND
		\end{itemize}
		\item Opcode \texttt{0xa}:
			\texttt{\textbf{orri} rA, \#simm}
		\begin{itemize}
			\item Note:  Bitwise OR
		\end{itemize}
		\item Opcode \texttt{0xb}:
			\texttt{\textbf{xori} rA, \#simm}
		\begin{itemize}
			\item Note:  Bitwise XOR
		\end{itemize}
		\item Opcode \texttt{0xc}:
			\texttt{\textbf{zeb} rA}
		\begin{itemize}
			\item Effect:  Set \texttt{rA[31:8]} to zero.
		\end{itemize}
		\item Opcode \texttt{0xd}:
			\texttt{\textbf{zeh} rA}
		\begin{itemize}
			\item Effect:  Set \texttt{rA[31:16]} to zero.
		\end{itemize}
		\item Opcode \texttt{0xe}:
			\texttt{\textbf{seb} rA}
		\begin{itemize}
			\item Effect:
				Sign-extend \texttt{rA[7:0]} to 32 bits, then copy that
				value to \texttt{rA}
		\end{itemize}
		\item Opcode \texttt{0xf}:
			\texttt{\textbf{seh} rA}
		\begin{itemize}
			\item Effect:
				Sign-extend \texttt{rA[15:0]} to 32 bits, then copy that
				value to \texttt{rA}
		\end{itemize}
	\end{itemize}

	\doublespacing

	\subsection{Instruction Group 2}
	The following encoding is used, with each character representing one
	bit:  \\
	\texttt{010f oooo cccc aaaa}, where

	\singlespacing
	\begin{itemize}
		\item \texttt{o} is the opcode
		\item \texttt{c} encodes register \texttt{rC}
		\item \texttt{a} encodes register \texttt{rA}
		\item \texttt{f} is encoded as \texttt{0} if this instruction
		cannot affect flags and encoded \texttt{1} if this instruction is
		permitted to affect flags.  Note that \texttt{cmp} is permitted to
		affect flags regardless of this bit.
	\end{itemize}
	\doublespacing

	Here is a list of instructions from this encoding group.

	\singlespacing
	\begin{itemize}
		\item Opcode \texttt{0x0}:
			\texttt{\textbf{add} rA, rC}
		\begin{itemize}
			\item Mnemonic for when flags not affected:  \texttt{add}
			\item Mnemonic for when flags affected:  \texttt{add.f}
			\item Affectable flags:
				\texttt{Z}, \texttt{C}, \texttt{V}, \texttt{N}
		\end{itemize}
		\item Opcode \texttt{0x1}:
			\texttt{\textbf{sub} rA, rC}
		\begin{itemize}
			\item Mnemonic for when flags not affected:  \texttt{sub}
			\item Mnemonic for when flags affected:  \texttt{sub.f}
			\item Affectable flags:
				\texttt{Z}, \texttt{C}, \texttt{V}, \texttt{N}
		\end{itemize}
		\item Opcode \texttt{0x2}:
			\texttt{\textbf{add} rA, sp, rC}
		\begin{itemize}
			\item Mnemonic for when flags not affected:  \texttt{add}
			\item Mnemonic for when flags affected:  \texttt{add.f}
			\item Affectable flags:
				\texttt{Z}, \texttt{C}, \texttt{V}, \texttt{N}
		\end{itemize}
		\item Opcode \texttt{0x3}
			\texttt{\textbf{add} rA, fp, rC}
		\begin{itemize}
			\item Mnemonic for when flags not affected:  \texttt{add}
			\item Mnemonic for when flags affected:  \texttt{add.f}
			\item Affectable flags:
				\texttt{Z}, \texttt{C}, \texttt{V}, \texttt{N}
		\end{itemize}
		\item Opcode \texttt{0x4}:
			\texttt{\textbf{cmp} rA, rC}
		\begin{itemize}
			\item Note:  Compare \texttt{rA} to \texttt{rC}.  \texttt{cmp}
			is \textit{always} able to affect flags, independent of the
			encoded \texttt{f} bit of the instruction.
			\item Affectable flags:
				\texttt{Z}, \texttt{C}, \texttt{V}, \texttt{N}
		\end{itemize}
		\item Opcode \texttt{0x5}:
			\texttt{\textbf{cpy} rA, rC}
		\begin{itemize}
			\item Mnemonic for when flags not affected:  \texttt{cpy}
			\item Mnemonic for when flags affected:  \texttt{cpy.f}
			\item Note:  Copy \texttt{rC} into \texttt{rA}
			\item Affectable flags:
				\texttt{Z}, \texttt{N}
		\end{itemize}
		\item Opcode \texttt{0x6}:
			\texttt{\textbf{lsl} rA, rC}
		\begin{itemize}
			\item Mnemonic for when flags not affected:  \texttt{lsl}
			\item Mnemonic for when flags affected:  \texttt{lsl.f}
			\item Note:  Logical shift left
			\item Affectable flags:
				\texttt{Z}, \texttt{N}
		\end{itemize}
		\item Opcode \texttt{0x7}:
			\texttt{\textbf{lsr} rA, rC}
		\begin{itemize}
			\item Mnemonic for when flags not affected:  \texttt{lsr}
			\item Mnemonic for when flags affected:  \texttt{lsr.f}
			\item Note:  Logical shift right
			\item Affectable flags:
				\texttt{Z}, \texttt{N}
		\end{itemize}
		\item Opcode \texttt{0x8}:
			\texttt{\textbf{asr} rA, rC}
		\begin{itemize}
			\item Mnemonic for when flags not affected:  \texttt{asr}
			\item Mnemonic for when flags affected:  \texttt{asr.f}
			\item Note:  Arithmetic shift right
			\item Affectable flags:
				\texttt{Z}, \texttt{N}
		\end{itemize}
		\item Opcode \texttt{0x9}:
			\texttt{\textbf{and} rA, rC}
		\begin{itemize}
			\item Mnemonic for when flags not affected:  \texttt{and}
			\item Mnemonic for when flags affected:  \texttt{and.f}
			\item Note:  Bitwise AND
			\item Affectable flags:
				\texttt{Z}, \texttt{N}
		\end{itemize}
		\item Opcode \texttt{0xa}:
			\texttt{\textbf{orr} rA, rC}
		\begin{itemize}
			\item Mnemonic for when flags not affected:  \texttt{orr}
			\item Mnemonic for when flags affected:  \texttt{orr.f}
			\item Note:  Bitwise OR
			\item Affectable flags:
				\texttt{Z}, \texttt{N}
		\end{itemize}
		\item Opcode \texttt{0xb}:
			\texttt{\textbf{xor} rA, rC}
		\begin{itemize}
			\item Mnemonic for when flags not affected:  \texttt{xor}
			\item Mnemonic for when flags affected:  \texttt{xor.f}
			\item Note:  Bitwise XOR
			\item Affectable flags:
				\texttt{Z}, \texttt{N}
		\end{itemize}
		\item Opcode \texttt{0xc}:
			\texttt{\textbf{adc} rA, rC}
		\begin{itemize}
			\item Mnemonic for when flags not affected:  \texttt{adc}
			\item Mnemonic for when flags affected:  \texttt{adc.f}
			\item Note:  Add with Carry
			\item Affectable flags:
				\texttt{Z}, \texttt{C}, \texttt{V}, \texttt{N}
		\end{itemize}
		\item Opcode \texttt{0xd}:
			\texttt{\textbf{sbc} rA, rC}
		\begin{itemize}
			\item Mnemonic for when flags not affected:  \texttt{sbc}
			\item Mnemonic for when flags affected:  \texttt{sbc.f}
			\item Note:  Subtract with Borrow
			\item Affectable flags:
				\texttt{Z}, \texttt{C}, \texttt{V}, \texttt{N}
		\end{itemize}
		%\item Opcode \texttt{0xe}:
		%	\texttt{\textbf{ludiv} rA, rC}
		%\begin{itemize}
		%	\item Effect:  performs a 64-bit by 64-bit unsigned division of
		%	\texttt{\{hi, lo\}} by \texttt{\{rA, rC\}}, storing 64-bit
		%	result in \texttt{\{hi, lo\}}.
		%	\item Note:  This instruction executes more quickly if
		%	\texttt{rA}'s value is \texttt{0x00000000}, i.e. if the
		%	operation is actually a 64-bit by 32-bit -> 64-bit unsigned
		%	divide.
		%\end{itemize}
		%\item Opcode \texttt{0xf}:
		%	\texttt{\textbf{lsdiv} rA, rB}
		%\begin{itemize}
		%	\item Effect:  performs a 64-bit by 64-bit signed division of
		%	\texttt{\{hi, lo\}} by \texttt{\{rA, rC\}}, storing 64-bit
		%	result in \texttt{\{hi, lo\}}.
		%	\item Note:  This instruction executes more quickly if
		%	\texttt{rA}'s value is either \texttt{0x00000000} or
		%	\texttt{0xffffffff},  i.e. if the operation is actually a
		%	64-bit by 32-bit -> 64-bit signed divide.
		%\end{itemize}
	\end{itemize}




	\doublespacing
	\subsection{Instruction Group 3:  Relative Branches}
	The following encoding is used, with each character representing one
	bit:  \\
	\texttt{100i iiii iiii oooo}, where

	\singlespacing
	\begin{itemize}
		\item \texttt{o} is the opcode
		\item \texttt{i} is a 9-bit sign-extended immediate, which can
		still be affected by \texttt{pre}, but when \texttt{pre} is the
		instruction before this one, only bits \texttt{[4:0]} of the
		immediate value encoded into this instruction will be used by the
		hardware, with the remaining bits of the immediate coming from
		however many \texttt{pre} instructions were used.  Multiple
		\texttt{pre} instructions can still form larger than 18-bit branch
		offsets.  This immediate is denoted \texttt{simm}.
	\end{itemize}
	\doublespacing

	Here is a list of instructions from this encoding group.

	\singlespacing
	\begin{itemize}
		%\item Opcode \texttt{0x0}:
		%	\texttt{\textbf{bnv} simm}
		%\begin{itemize}
		%	\item Effect:
		%		\texttt{if (0) pc <= pc + simm + 2;} This is a NOP.
		%\end{itemize}
		\item Opcode \texttt{0x0}:
			\texttt{\textbf{bl} \#simm}
		\begin{itemize}
			\item Effect:
				\texttt{lr <= pc + 2; pc <= pc + simm + 2;}
		\end{itemize}

		\item Opcode \texttt{0x1}:
			\texttt{\textbf{bra} \#simm}
		\begin{itemize}
			\item Effect:
				\texttt{pc <= pc + simm + 2;}
		\end{itemize}

		\item Opcode \texttt{0x2}:
			\texttt{\textbf{bne} \#simm}
		\begin{itemize}
			\item Effect:
				\texttt{if (!flags.Z) pc <= pc + simm + 2;}
		\end{itemize}

		\item Opcode \texttt{0x3}:
			\texttt{\textbf{beq} \#simm}
		\begin{itemize}
			\item Effect:
				\texttt{if (flags.Z) pc <= pc + simm + 2;}
		\end{itemize}

		\item Opcode \texttt{0x4}:
			\texttt{\textbf{bpl} \#simm}
		\begin{itemize}
			\item Effect:
				\texttt{if (!flags.N) pc <= pc + simm + 2;}
		\end{itemize}

		\item Opcode \texttt{0x5}:
			\texttt{\textbf{bmi} \#simm}
		\begin{itemize}
			\item Effect:
				\texttt{if (flags.N) pc <= pc + simm + 2;}
		\end{itemize}

		\item Opcode \texttt{0x6}:
			\texttt{\textbf{bvc} \#simm}
		\begin{itemize}
			\item Effect:
				\texttt{if (!flags.V) pc <= pc + simm + 2;}
		\end{itemize}

		\item Opcode \texttt{0x7}:
			\texttt{\textbf{bvs} \#simm}
		\begin{itemize}
			\item Effect:
				\texttt{if (flags.V) pc <= pc + simm + 2;}
		\end{itemize}

		\item Opcode \texttt{0x8}:
			\texttt{\textbf{bgeu} \#simm}
		\begin{itemize}
			\item Effect:
				\texttt{if (flags.C) pc <= pc + simm + 2;}
		\end{itemize}

		\item Opcode \texttt{0x9}:
			\texttt{\textbf{bltu} \#simm}
		\begin{itemize}
			\item Effect:
				\texttt{if (!flags.C) pc <= pc + simm + 2;}
		\end{itemize}

		\item Opcode \texttt{0xa}:
			\texttt{\textbf{bgtu} \#simm}
		\begin{itemize}
			\item Effect:
				\texttt{if (flags.C AND (!flags.Z)) pc <= pc + simm + 2;}
		\end{itemize}

		\item Opcode \texttt{0xb}:
			\texttt{\textbf{bleu} \#simm}
		\begin{itemize}
			\item Effect:
				\texttt{if ((!flags.C) OR flags.Z) pc <= pc + simm + 2;}
		\end{itemize}

		\item Opcode \texttt{0xc}:
			\texttt{\textbf{bges} \#simm}
		\begin{itemize}
			\item Effect:
				\texttt{if (!(flags.N XOR flags.V)) pc <= pc + simm + 2;}
		\end{itemize}

		\item Opcode \texttt{0xd}:
			\texttt{\textbf{blts} \#simm}
		\begin{itemize}
			\item Effect:
				\texttt{if (flags.N XOR flags.V) pc <= pc + simm + 2;}
		\end{itemize}

		\item Opcode \texttt{0xe}:
			\texttt{\textbf{bgts} \#simm}
		\begin{itemize}
			\item Effect:
				\texttt{if ((!(flags.N XOR flags.V)) AND (!flags.Z))
					pc <= pc + simm + 2;}
		\end{itemize}

		\item Opcode \texttt{0xf}:
			\texttt{\textbf{bles} \#simm}
		\begin{itemize}
			\item Effect:
				\texttt{if ((flags.N XOR flags.V) OR flags.Z)
					pc <= pc + simm + 2;}
		\end{itemize}
	\end{itemize}

	\doublespacing
	\subsection{Instruction Group 4}
	The following encoding is used, with each character representing one
	bit:  \\
	\texttt{100o oooo cccc aaaa}, where

	\singlespacing
	\begin{itemize}
		\item \texttt{o} is the opcode
		\item \texttt{c} encodes register \texttt{rC} or register
		\texttt{sC}, where \texttt{sC} is one of \texttt{hi}, \texttt{lo},
		\texttt{flags}, \texttt{ira}, \texttt{ids}, or \texttt{ie}
		\item \texttt{a} encodes register \texttt{rA} or register
		\texttt{sA}, where \texttt{sA} is one of \texttt{hi}, \texttt{lo},
		\texttt{flags}, \texttt{ira}, \texttt{ids}, or \texttt{ie}
	\end{itemize}
	\doublespacing

	Here is a list of instructions from this encoding group.

	\singlespacing
	\begin{itemize}
		\item Opcode \texttt{0x0}:
			\texttt{\textbf{jl} rA}
		\begin{itemize}
			\item Effect:  \texttt{lr <= pc + 2; pc <= rA;}
		\end{itemize}
		\item Opcode \texttt{0x1}:
			\texttt{\textbf{jmp} rA}
		\begin{itemize}
			\item Effect:  \texttt{pc <= rA;}
		\end{itemize}
		\item Opcode \texttt{0x2}:
			\texttt{\textbf{jmp} rA, rC}
		\begin{itemize}
			\item Effect:  \texttt{pc <= (rA + rC);}
		\end{itemize}
		\item Opcode \texttt{0x3}:
			\texttt{\textbf{jmp} ira}
		\begin{itemize}
			\item Effect:  \texttt{pc <= ira;}
		\end{itemize}
		\item Opcode \texttt{0x4}:
			\texttt{\textbf{reti}}
		\begin{itemize}
			\item Effect:  enables interrupts (by copying \texttt{1} into
			\texttt{ie}) and performs \texttt{pc <= ira;}
		\end{itemize}
		\item Opcode \texttt{0x5}:
			\texttt{\textbf{cpy} rA, sC}
		\begin{itemize}
			\item Effect:  \texttt{rA <= sC;}
		\end{itemize}
		\item Opcode \texttt{0x6}:
			\texttt{\textbf{cpy} sA, rC}
		\begin{itemize}
			\item Effect:  \texttt{sA <= rC;}
		\end{itemize}
		\item Opcode \texttt{0x7}:
			\texttt{\textbf{ei}}
		\begin{itemize}
			\item Effect:  copy \texttt{1} into \texttt{ie}.
		\end{itemize}
		\item Opcode \texttt{0x8}:
			\texttt{\textbf{di}}
		\begin{itemize}
			\item Effect:  copy \texttt{0} into \texttt{ie}.
		\end{itemize}
		\item Opcode \texttt{0x9}:
			\texttt{\textbf{push} rA, rC}
		\begin{itemize}
			\item Effect:  pushes \texttt{rA} onto the stack, using
			\texttt{rC} as the stack pointer, post-decrementing
			\texttt{rC}.
		\end{itemize}
		\item Opcode \texttt{0xa}:
			\texttt{\textbf{pop} rA, rC}
		\begin{itemize}
			\item Effect:  pops \texttt{rA} off the stack, using
			\texttt{rC} as the stack pointer, pre-incrementing \texttt{rC}.
		\end{itemize}
		\item Opcode \texttt{0xb}:
			\texttt{\textbf{push} sA, rC}
		\begin{itemize}
			\item Effect:  pushes \texttt{sA} onto the stack, using
			\texttt{rC} as the stack pointer, post-decrementing
			\texttt{rC}.
			\item Note that \texttt{sA} is considered to be 32-bit for the
			purpose of the store to memory and decrementing \texttt{rC},
			even if \texttt{sA} is \texttt{flags} or \texttt{ie}.
		\end{itemize}
		\item Opcode \texttt{0xc}:
			\texttt{\textbf{pop} sA, rC}
		\begin{itemize}
			\item Effect:  pops \texttt{sA} off the stack, using
			\texttt{rC} as the stack pointer, pre-incrementing \texttt{rC}.
			\item Note that \texttt{sA} is considered to be 32-bit for the
			purpose of the load from memory and incrementing \texttt{rC},
			even if \texttt{sA} is \texttt{flags} or \texttt{ie}.
		\end{itemize}
		\item Opcode \texttt{0xd}:
			\texttt{\textbf{index} rA}
		\begin{itemize}
			\item Effect:  Performs \texttt{<index\_reg> <= rA;} and stores
			that \texttt{index} is in effect.
			\item Note:  If \texttt{index} is in effect and the current
			instruction is \texttt{index}, the current instruction will be
			treated as a NOP, and \texttt{index} will stop being in effect.
			\item Note:  \texttt{pre} and \texttt{index} can be combined
			with one another (though this is only useful for \texttt{ldr}
			and \texttt{str}).
			\item Note:  A non-\texttt{pre} instruction following
			\texttt{index} will store that that \texttt{index} is not in
			effect any more.  (It will also store that \texttt{pre} is not
			in effect any more).
			\item Note:  If \texttt{index} is in effect, the current
			instruction cannot be interrupted.
			\item Note:  Any time \texttt{index} stops being in effect,
			\texttt{pre} will stop being in effect as well.
		\end{itemize}
		\item Opcode \texttt{0xe}:
			\texttt{\textbf{mul} rA, rC}
		\begin{itemize}
			\item Effect:  \texttt{rA <= rA * rC;}
		\end{itemize}
		\item Opcode \texttt{0xf}:
			\texttt{\textbf{udiv} rA, rC}
		\begin{itemize}
			\item Effect:  \texttt{rA <= u32(rA) / u32(rC);}
		\end{itemize}
		\item Opcode \texttt{0x10}:
			\texttt{\textbf{sdiv} rA, rC}
		\begin{itemize}
			\item Effect:  \texttt{rA <= s32(rA) / s32(rC);}
		\end{itemize}
		\item Opcode \texttt{0x11}:
			\texttt{\textbf{umod} rA, rC}
		\begin{itemize}
			\item Effect:  \texttt{rA <= u32(rA) \% u32(rC);}
		\end{itemize}
		\item Opcode \texttt{0x12}:
			\texttt{\textbf{smod} rA, rC}
		\begin{itemize}
			\item Effect:  \texttt{rA <= s32(rA) \% s32(rC);}
		\end{itemize}
		\item Opcode \texttt{0x13}:
			\texttt{\textbf{lumul} rA, rC}
		\begin{itemize}
			\item Effect:  This instruction multiplies \texttt{rA} by
			\texttt{rC}, performing an unsigned 32-bit by 32-bit -> 64-bit
			multiply, storing result in \texttt{\{hi, lo\}}.
		\end{itemize}
		\item Opcode \texttt{0x14}:
			\texttt{\textbf{lsmul} rA, rC}
		\begin{itemize}
			\item Effect:  This instruction multiplies \texttt{rA} by
			\texttt{rC}, performing a signed 32-bit by 32-bit -> 64-bit
			multiply, storing result in \texttt{\{hi, lo\}}.
		\end{itemize}
		\item Opcode \texttt{0x15}:
			\texttt{\textbf{ludiv} rA, rC}
		\begin{itemize}
			\item Effect:  performs a 64-bit by 64-bit unsigned division of
			\texttt{\{hi, lo\}} by \texttt{\{rA, rC\}}, storing 64-bit
			result in \texttt{\{hi, lo\}}.
			\item Note:  This instruction executes more quickly if
			\texttt{rA}'s value is \texttt{0x00000000}, i.e. if the
			operation is actually a 64-bit by 32-bit -> 64-bit unsigned
			divide.
		\end{itemize}
		\item Opcode \texttt{0x16}:
			\texttt{\textbf{lsdiv} rA, rC}
		\begin{itemize}
			\item Effect:  performs a 64-bit by 64-bit signed division of
			\texttt{\{hi, lo\}} by \texttt{\{rA, rC\}}, storing 64-bit
			result in \texttt{\{hi, lo\}}.
			%\item Note:  This instruction executes more quickly if
			%\texttt{rA}'s value is either \texttt{0x00000000} or
			%\texttt{0xffffffff},  i.e. if the operation is actually a
			%64-bit by 32-bit -> 64-bit signed divide.
			\item Note:  This instruction executes more quickly if
			\texttt{rA} is equal to bits \texttt{[63:32]} of
			\texttt{sign\_extend\_to\_64(rC)}.
		\end{itemize}
		\item Opcode \texttt{0x17}:
			\texttt{\textbf{lumod} rA, rC}
		\begin{itemize}
			\item Effect:  performs a 64-bit by 64-bit unsigned modulo of
			\texttt{\{hi, lo\}} by \texttt{\{rA, rC\}}, storing 64-bit
			result in \texttt{\{hi, lo\}}.
			\item Note:  This instruction executes more quickly if
			\texttt{rA}'s value is \texttt{0x00000000}, i.e. if the
			operation is actually a 64-bit by 32-bit -> 64-bit unsigned
			modulo.
		\end{itemize}
		\item Opcode \texttt{0x18}:
			\texttt{\textbf{lsmod} rA, rC}
		\begin{itemize}
			\item Effect:  performs a 64-bit by 64-bit signed modulo of
			\texttt{\{hi, lo\}} by \texttt{\{rA, rC\}}, storing 64-bit
			result in \texttt{\{hi, lo\}}.
			\item Note:  This instruction executes more quickly if
			\texttt{rA} is equal to bits \texttt{[63:32]} of
			\texttt{sign\_extend\_to\_64(rC)}.
		\end{itemize}
		\item Opcode \texttt{0x19}:
			\texttt{\textbf{ldub} rA, [rC]}
		\begin{itemize}
			\item Effect:  Load an 8-bit value from memory at address
			computed as \texttt{rC + <index\_reg>}, zero-extend
			the 8-bit value to 32 bits, then put the zero-extended 32-bit
			value into \texttt{rA}.
			\item The \texttt{<index\_reg>} value is guaranteed to be zero
			unless an \texttt{index} is in effect.
			\item Shorthand for having the assembler insert an
			\texttt{index rB} instruction before this one:
				\texttt{ldub rA, [rC, rB]}
		\end{itemize}
		\item Opcode \texttt{0x1a}:
			\texttt{\textbf{ldsb} rA, [rC]}
		\begin{itemize}
			\item Effect:  Load an 8-bit value from memory at address
			computed as \texttt{rC + <index\_reg>}, sign-extend
			the 8-bit value to 32 bits, then put the sign-extended 32-bit
			value into \texttt{rA}.
			\item The \texttt{<index\_reg>} value is guaranteed to be zero
			unless an \texttt{index} is in effect.
			\item Shorthand for having the assembler insert an
			\texttt{index rB} instruction before this one:
				\texttt{ldsb rA, [rC, rB]}
		\end{itemize}
		\item Opcode \texttt{0x1b}:
			\texttt{\textbf{lduh} rA, [rC]}
		\begin{itemize}
			\item Effect:  Load a 16-bit value from memory at address
			computed as \texttt{rC + <index\_reg>}, zero-extend
			the 16-bit value to 32 bits, then put the zero-extended 32-bit
			value into \texttt{rA}.
			\item The \texttt{<index\_reg>} value is guaranteed to be zero
			unless an \texttt{index} is in effect.
			\item Shorthand for having the assembler insert an
			\texttt{index rB} instruction before this one:
				\texttt{lduh rA, [rC, rB]}
		\end{itemize}
		\item Opcode \texttt{0x1c}:
			\texttt{\textbf{ldsh} rA, [rC]}
		\begin{itemize}
			\item Effect:  Load a 16-bit value from memory at address
			computed as \texttt{rC + <index\_reg>}, sign-extend
			the 16-bit value to 32 bits, then put the zero-extended 32-bit
			value into \texttt{rA}.
			\item The \texttt{<index\_reg>} value is guaranteed to be zero
			unless an \texttt{index} is in effect.
			\item Shorthand for having the assembler insert an
			\texttt{index rB} instruction before this one:
				\texttt{ldsh rA, [rC, rB]}
		\end{itemize}
		\item Opcode \texttt{0x1d}:
			\texttt{\textbf{stb} rA, [rC]}
		\begin{itemize}
			\item Effect:  Store \texttt{rA[7:0]} to memory at the address
			computed as \texttt{rC + <index\_reg>}.
			\item The \texttt{<index\_reg>} value is guaranteed to be zero
			unless an \texttt{index} is in effect.
			\item Shorthand for having the assembler insert an
			\texttt{index rB} instruction before this one:
				\texttt{stb rA, [rC, rB]}
		\end{itemize}
		\item Opcode \texttt{0x1e}:
			\texttt{\textbf{sth} rA, [rC]}
		\begin{itemize}
			\item Effect:  Store \texttt{rA[15:0]} to memory at the address
			computed as \texttt{rC + <index\_reg>}.
			\item The \texttt{<index\_reg>} value is guaranteed to be zero
			unless an \texttt{index} is in effect.
			\item Shorthand for having the assembler insert an
			\texttt{index rB} instruction before this one:
				\texttt{sth rA, [rC, rB]}
		\end{itemize}

	\end{itemize}

	\doublespacing
	\subsection{Instruction Group 5:  Immediate Indexed Load}
	The following encoding is used, with each character representing one
	bit:  \\
	\texttt{101i iiii cccc aaaa}, where

	\singlespacing
	\begin{itemize}
		\item \texttt{i} is a 5-bit sign-extended immediate, which can
		be expanded by \texttt{pre}, and is denoted \texttt{simm}
		\item \texttt{c} encodes register \texttt{rC}
		\item \texttt{a} encodes register \texttt{rA}
	\end{itemize}
	\doublespacing

	The one instruction from this encoding group is
	\texttt{\textbf{ldr} rA, [rC, \#simm]}.
	This is a 32-bit load into \texttt{rA}, where the effective address to
	load from is computed as \texttt{rC + <index\_reg> + simm}, using the
	sign-extended form of \texttt{simm}.

	The \texttt{<index\_reg>} value is guaranteed to be zero unless an
	\texttt{index} is in effect.

	Shorthand for having the assembler insert an \texttt{index rB}
	instruction before this one: \texttt{ldr rA, [rC, rB, simm]}

	%The \texttt{<index\_reg>} value is zero in all cases except
	%the previous instruction was \texttt{index index\_reg} and if
	%\texttt{index\_reg}'s value was non-zero. 

	\subsection{Instruction Group 6:  Immediate Indexed Store}
	The following encoding is used, with each character representing one
	bit:  \\
	\texttt{110i iiii cccc aaaa}, where

	\singlespacing
	\begin{itemize}
		\item \texttt{i} is a 5-bit sign-extended immediate, which can
		be expanded by \texttt{pre}  
		\item \texttt{c} encodes register \texttt{rC}
		\item \texttt{a} encodes register \texttt{rA}
	\end{itemize}
	\doublespacing

	The one instruction from this encoding group is
	\texttt{\textbf{str} rA, [rC, \#simm]}.
	This is a 32-bit store of \texttt{rA}, where the effective address to
	store to is computed as \texttt{<index\_reg> + rC + simm}, using the
	sign-extended form of \texttt{simm}.

	The \texttt{<index\_reg>} value is guaranteed to be zero unless an
	\texttt{index} is in effect.

	Shorthand for having the assembler insert an \texttt{index rB}
	instruction before this one: \texttt{str rA, [rC, rB, \#simm]}

	%The \texttt{index\_reg} value is zero in all cases except
	%the previous instruction was \texttt{index index\_reg} and if
	%\texttt{index\_reg}'s value was non-zero. 

	%\subsection{Instruction Group 7.0:  Other Loads and Stores}
	%The following encoding is used, with each character representing one
	%bit:  \\
	%\texttt{1110 cccc aaaa 00oo}, where

	%\singlespacing
	%\begin{itemize}
	%	\item \texttt{c} encodes register \texttt{rC}
	%	\item \texttt{a} encodes register \texttt{rA}
	%	\item \texttt{o} is the opcode
	%\end{itemize}
	%\doublespacing

	%Here is a list of instructions from this encoding group.

	%\singlespacing
	%\begin{itemize}
	%	\item Opcode \texttt{0x0}:
	%		\texttt{\textbf{ldb} rA, [rC]}
	%	\begin{itemize}
	%		\item Effect:  Load a byte from memory at address computed as
	%		just the value of \texttt{rC}, zero-extend the byte to 32 bits,
	%		then put the zero-extended 32-bit value into \texttt{rA}.
	%	\end{itemize}
	%	\item Opcode \texttt{0x1}:
	%		\texttt{\textbf{ldh} rA, [rC]}
	%	\begin{itemize}
	%		\item Effect:  Load a 16-bit value from memory at address
	%		computed as just the value of \texttt{rC}, zero-extend the
	%		16-bit value to 32 bits, then put the zero-extended 32-bit
	%		value into \texttt{rA}.
	%	\end{itemize}
	%	\item Opcode \texttt{0x2}:
	%		\texttt{\textbf{stb} rA, [rC]}
	%	\begin{itemize}
	%		\item Effect:  Store \texttt{rA[7:0]} to memory at the address
	%		computed as just the value of \texttt{rC}.
	%	\end{itemize}
	%	\item Opcode \texttt{0x3}:
	%		\texttt{\textbf{sth} rA, [rC]}
	%	\begin{itemize}
	%		\item Effect:  Store \texttt{rA[15:0]} to memory at the address
	%		computed as just the value of \texttt{rC}.
	%	\end{itemize}
	%\end{itemize}
	%\doublespacing


	%\doublespacing
	%\subsection{Instruction Group 3:  Jumps}
	%The following encoding is used, with each character representing one
	%bit:  \\
	%\texttt{011o ooo0 0000 bbbb}, where

	%\singlespacing
	%\begin{itemize}
	%	\item \texttt{o} is the opcode
	%	\item \texttt{b} encodes register \texttt{rB}
	%\end{itemize}

	%\doublespacing

	%%At the assembly language level, \texttt{pick} is simply
	%%\texttt{pick rB}, but as with \texttt{pre}, it isn't typical to use
	%%this instruction directly in assembly language source code.  Instead,
	%%when an instruction is typed as \texttt{instr rA, rB, operand}
	%%\textit{and} when \texttt{rA} is not the same register as \texttt{rB},
	%%a \texttt{pick rB} instruction will be inserted by the assembler before
	%%the instruction in the source code.  As mentioned before, when the
	%%instruction previous to the current one is either a \texttt{pre} or a
	%%\texttt{pick}, interrupts will not be serviced.  Note that since
	%%\texttt{pre}'s 13-bit partial immediate only affects the instruction
	%%following the \texttt{pre}, only a 5-bit signed immediate can be used
	%%with \texttt{pick} and an instruction from group 1.  For all
	%%instructions, the default value of \texttt{rB} is \texttt{rA}, except
	%%when the previous instruction was \texttt{pick}.

	%Here is the list of instructions from this encoding group.

	%\singlespacing
	%\begin{itemize}
	%	\item Opcode \texttt{0x0}:
	%		\texttt{\textbf{jnv} rB}
	%	\begin{itemize}
	%		\item Effect:
	%			\texttt{if (0) pc <= rB;} This is a NOP.
	%	\end{itemize}

	%	\item Opcode \texttt{0x1}:
	%		\texttt{\textbf{jmp} rB}
	%	\begin{itemize}
	%		\item Effect:
	%			\texttt{if (1) pc <= rB;}
	%	\end{itemize}

	%	\item Opcode \texttt{0x2}:
	%		\texttt{\textbf{jne} rB}
	%	\begin{itemize}
	%		\item Effect:
	%			\texttt{if (!flags.Z) pc <= rB;}
	%	\end{itemize}

	%	\item Opcode \texttt{0x3}:
	%		\texttt{\textbf{jeq} rB}
	%	\begin{itemize}
	%		\item Effect:
	%			\texttt{if (flags.Z) pc <= rB;}
	%	\end{itemize}

	%	\item Opcode \texttt{0x4}:
	%		\texttt{\textbf{jpl} rB}
	%	\begin{itemize}
	%		\item Effect:
	%			\texttt{if (!flags.N) pc <= rB;}
	%	\end{itemize}

	%	\item Opcode \texttt{0x5}:
	%		\texttt{\textbf{jmi} rB}
	%	\begin{itemize}
	%		\item Effect:
	%			\texttt{if (flags.N) pc <= rB;}
	%	\end{itemize}

	%	\item Opcode \texttt{0x6}:
	%		\texttt{\textbf{jvc} rB}
	%	\begin{itemize}
	%		\item Effect:
	%			\texttt{if (!flags.V) pc <= rB;}
	%	\end{itemize}

	%	\item Opcode \texttt{0x7}:
	%		\texttt{\textbf{jvs} rB}
	%	\begin{itemize}
	%		\item Effect:
	%			\texttt{if (flags.V) pc <= rB;}
	%	\end{itemize}

	%	\item Opcode \texttt{0x8}:
	%		\texttt{\textbf{jgeu} rB}
	%	\begin{itemize}
	%		\item Effect:
	%			\texttt{if (flags.C) pc <= rB;}
	%	\end{itemize}

	%	\item Opcode \texttt{0x9}:
	%		\texttt{\textbf{jltu} rB}
	%	\begin{itemize}
	%		\item Effect:
	%			\texttt{if (!flags.C) pc <= rB;}
	%	\end{itemize}

	%	\item Opcode \texttt{0xa}:
	%		\texttt{\textbf{jgtu} rB}
	%	\begin{itemize}
	%		\item Effect:
	%			\texttt{if (flags.C AND (!flags.Z)) pc <= rB;}
	%	\end{itemize}

	%	\item Opcode \texttt{0xb}:
	%		\texttt{\textbf{jleu} rB}
	%	\begin{itemize}
	%		\item Effect:
	%			\texttt{if ((!flags.C) OR flags.Z) pc <= rB;}
	%	\end{itemize}

	%	\item Opcode \texttt{0xc}:
	%		\texttt{\textbf{jges} rB}
	%	\begin{itemize}
	%		\item Effect:
	%			\texttt{if (!(flags.N XOR flags.V)) pc <= rB;}
	%	\end{itemize}

	%	\item Opcode \texttt{0xd}:
	%		\texttt{\textbf{jlts} rB}
	%	\begin{itemize}
	%		\item Effect:
	%			\texttt{if (flags.N XOR flags.V) pc <= rB;}
	%	\end{itemize}

	%	\item Opcode \texttt{0xe}:
	%		\texttt{\textbf{jgts} rB}
	%	\begin{itemize}
	%		\item Effect:
	%			\texttt{if ((!(flags.N XOR flags.V)) AND (!flags.Z))
	%				pc <= rB;}
	%	\end{itemize}

	%	\item Opcode \texttt{0xf}:
	%		\texttt{\textbf{jles} rB}
	%	\begin{itemize}
	%		\item Effect:
	%			\texttt{if ((flags.N XOR flags.V) OR flags.Z)
	%				pc <= rB;}
	%	\end{itemize}
	%\end{itemize}

	%\doublespacing

	%\subsection{Instruction Group 4:  Relative Branches}
	%The following encoding is used, with each character representing one
	%bit:  \\
	%\texttt{100o oooi iiii iiii}, where

	%\singlespacing
	%\begin{itemize}
	%	\item \texttt{o} is the opcode
	%	\item \texttt{i} is a 9-bit sign-extended immediate, which can
	%	still be affected by \texttt{pre}, but when \texttt{pre} is the
	%	instruction before this one, only bits \texttt{[4:0]} of the
	%	immediate value encoded into this instruction will be used by the
	%	hardware, with the remaining bits of the immediate coming from
	%	however many \texttt{pre} instructions were used.  Multiple
	%	\texttt{pre} instructions can still form larger than 18-bit branch
	%	offsets.
	%\end{itemize}
	%\doublespacing

	%Here is a list of instructions from this encoding group.

	%\singlespacing
	%\begin{itemize}
	%	%\item Opcode \texttt{0x0}:
	%	%	\texttt{\textbf{bnv} simm}
	%	%\begin{itemize}
	%	%	\item Effect:
	%	%		\texttt{if (0) pc <= pc + simm + 2;} This is a NOP.
	%	%\end{itemize}
	%	\item Opcode \texttt{0x0}:
	%		\texttt{\textbf{bnv} simm}
	%	\begin{itemize}
	%		\item Effect:
	%			\texttt{if (0) pc <= pc + simm + 2;} This is a NOP.
	%	\end{itemize}

	%	\item Opcode \texttt{0x1}:
	%		\texttt{\textbf{bra} simm}
	%	\begin{itemize}
	%		\item Effect:
	%			\texttt{if (1) pc <= pc + simm + 2;}
	%	\end{itemize}

	%	\item Opcode \texttt{0x2}:
	%		\texttt{\textbf{bne} simm}
	%	\begin{itemize}
	%		\item Effect:
	%			\texttt{if (!flags.Z) pc <= pc + simm + 2;}
	%	\end{itemize}

	%	\item Opcode \texttt{0x3}:
	%		\texttt{\textbf{beq} simm}
	%	\begin{itemize}
	%		\item Effect:
	%			\texttt{if (flags.Z) pc <= pc + simm + 2;}
	%	\end{itemize}

	%	\item Opcode \texttt{0x4}:
	%		\texttt{\textbf{bpl} simm}
	%	\begin{itemize}
	%		\item Effect:
	%			\texttt{if (!flags.N) pc <= pc + simm + 2;}
	%	\end{itemize}

	%	\item Opcode \texttt{0x5}:
	%		\texttt{\textbf{bmi} simm}
	%	\begin{itemize}
	%		\item Effect:
	%			\texttt{if (flags.N) pc <= pc + simm + 2;}
	%	\end{itemize}

	%	\item Opcode \texttt{0x6}:
	%		\texttt{\textbf{bvc} simm}
	%	\begin{itemize}
	%		\item Effect:
	%			\texttt{if (!flags.V) pc <= pc + simm + 2;}
	%	\end{itemize}

	%	\item Opcode \texttt{0x7}:
	%		\texttt{\textbf{bvs} simm}
	%	\begin{itemize}
	%		\item Effect:
	%			\texttt{if (flags.V) pc <= pc + simm + 2;}
	%	\end{itemize}

	%	\item Opcode \texttt{0x8}:
	%		\texttt{\textbf{bgeu} simm}
	%	\begin{itemize}
	%		\item Effect:
	%			\texttt{if (flags.C) pc <= pc + simm + 2;}
	%	\end{itemize}

	%	\item Opcode \texttt{0x9}:
	%		\texttt{\textbf{bltu} simm}
	%	\begin{itemize}
	%		\item Effect:
	%			\texttt{if (!flags.C) pc <= pc + simm + 2;}
	%	\end{itemize}

	%	\item Opcode \texttt{0xa}:
	%		\texttt{\textbf{bgtu} simm}
	%	\begin{itemize}
	%		\item Effect:
	%			\texttt{if (flags.C AND (!flags.Z)) pc <= pc + simm + 2;}
	%	\end{itemize}

	%	\item Opcode \texttt{0xb}:
	%		\texttt{\textbf{bleu} simm}
	%	\begin{itemize}
	%		\item Effect:
	%			\texttt{if ((!flags.C) OR flags.Z) pc <= pc + simm + 2;}
	%	\end{itemize}

	%	\item Opcode \texttt{0xc}:
	%		\texttt{\textbf{bges} simm}
	%	\begin{itemize}
	%		\item Effect:
	%			\texttt{if (!(flags.N XOR flags.V)) pc <= pc + simm + 2;}
	%	\end{itemize}

	%	\item Opcode \texttt{0xd}:
	%		\texttt{\textbf{blts} simm}
	%	\begin{itemize}
	%		\item Effect:
	%			\texttt{if (flags.N XOR flags.V) pc <= pc + simm + 2;}
	%	\end{itemize}

	%	\item Opcode \texttt{0xe}:
	%		\texttt{\textbf{bgts} simm}
	%	\begin{itemize}
	%		\item Effect:
	%			\texttt{if ((!(flags.N XOR flags.V)) AND (!flags.Z))
	%				pc <= pc + simm + 2;}
	%	\end{itemize}

	%	\item Opcode \texttt{0xf}:
	%		\texttt{\textbf{bles} simm}
	%	\begin{itemize}
	%		\item Effect:
	%			\texttt{if ((flags.N XOR flags.V) OR flags.Z)
	%				pc <= pc + simm + 2;}
	%	\end{itemize}
	%\end{itemize}

	%\doublespacing
	%\subsection{Instruction Group 5:  Indexed Load}
	%The following encoding is used, with each character representing one
	%bit:  \\
	%\texttt{101i cccc aaaa iiii}, where

	%\singlespacing
	%\begin{itemize}
	%	\item \texttt{i} is a 5-bit sign-extended immediate, which can
	%	be expanded by \texttt{pre}  
	%	\item \texttt{c} encodes register \texttt{rC}
	%	\item \texttt{a} encodes register \texttt{rA}
	%\end{itemize}
	%\doublespacing

	%The one instruction from this encoding group is
	%\texttt{\textbf{ldr} rA, [rC, simm]}.
	%This is a 32-bit load into \texttt{rA}, where the effective address to
	%load from is computed as \texttt{rC + simm}, using the sign-extended
	%form of \texttt{simm}.
	%%Note that \texttt{pick} does not affect this instruction in any way.

	%\subsection{Instruction Group 6:  Indexed Store}
	%The following encoding is used, with each character representing one
	%bit:  \\
	%\texttt{110i cccc aaaa iiii}, where

	%\singlespacing
	%\begin{itemize}
	%	\item \texttt{i} is a 5-bit sign-extended immediate, which can
	%	be expanded by \texttt{pre}  
	%	\item \texttt{c} encodes register \texttt{rC}
	%	\item \texttt{a} encodes register \texttt{rA}
	%\end{itemize}
	%\doublespacing

	%The one instruction from this encoding group is
	%\texttt{\textbf{str} rA, [rC, simm]}.
	%This is a 32-bit store of \texttt{rA}, where the effective address to
	%store to is computed as \texttt{rC + simm}, using the sign-extended
	%form of \texttt{simm}.
	%%Note that \texttt{pick} does not affect this instruction in any way.

	%\subsection{Instruction Group 7.0:  Other Loads and Stores}
	%The following encoding is used, with each character representing one
	%bit:  \\
	%\texttt{1110 cccc aaaa 00oo}, where

	%\singlespacing
	%\begin{itemize}
	%	\item \texttt{c} encodes register \texttt{rC}
	%	\item \texttt{a} encodes register \texttt{rA}
	%	\item \texttt{o} is the opcode
	%\end{itemize}
	%\doublespacing

	%Here is a list of instructions from this encoding group.

	%\singlespacing
	%\begin{itemize}
	%	\item Opcode \texttt{0x0}:
	%		\texttt{\textbf{ldb} rA, [rC]}
	%	\begin{itemize}
	%		\item Effect:  Load a byte from memory at address computed as
	%		just the value of \texttt{rC}, zero-extend the byte to 32 bits,
	%		then put the zero-extended 32-bit value into \texttt{rA}.
	%	\end{itemize}
	%	\item Opcode \texttt{0x1}:
	%		\texttt{\textbf{ldh} rA, [rC]}
	%	\begin{itemize}
	%		\item Effect:  Load a 16-bit value from memory at address
	%		computed as just the value of \texttt{rC}, zero-extend the
	%		16-bit value to 32 bits, then put the zero-extended 32-bit
	%		value into \texttt{rA}.
	%	\end{itemize}
	%	\item Opcode \texttt{0x2}:
	%		\texttt{\textbf{stb} rA, [rC]}
	%	\begin{itemize}
	%		\item Effect:  Store \texttt{rA[7:0]} to memory at the address
	%		computed as just the value of \texttt{rC}.
	%	\end{itemize}
	%	\item Opcode \texttt{0x3}:
	%		\texttt{\textbf{sth} rA, [rC]}
	%	\begin{itemize}
	%		\item Effect:  Store \texttt{rA[15:0]} to memory at the address
	%		computed as just the value of \texttt{rC}.
	%	\end{itemize}
	%\end{itemize}
	%\doublespacing

	%\subsection{Instruction Group 7.1:  Copying to/from Special Registers}
	%The following encoding is used, with each character representing one
	%bit:  \\
	%\texttt{1110 oooo aaaa 0100}, where

	%\singlespacing
	%\begin{itemize}
	%	\item \texttt{o} is the opcode
	%	\item \texttt{a} encodes register \texttt{rA}
	%\end{itemize}
	%\doublespacing

	%Here is a list of instructions from this encoding group.

	%\singlespacing
	%\begin{itemize}
	%	\item Opcode \texttt{0x0}:
	%		\texttt{\textbf{cpyspec} hi, rA}
	%	\item Opcode \texttt{0x1}:
	%		\texttt{\textbf{cpyspec} rA, hi}
	%	\item Opcode \texttt{0x2}:
	%		\texttt{\textbf{cpyspec} lo, rA}
	%	\item Opcode \texttt{0x3}:
	%		\texttt{\textbf{cpyspec} rA, lo}
	%	\item Opcode \texttt{0x4}:
	%		\texttt{\textbf{cpyspec} flags, rA}
	%	\begin{itemize}
	%		\item Effect:  copy \texttt{rA[3:0]} to \texttt{flags}.
	%	\end{itemize}
	%	\item Opcode \texttt{0x5}:
	%		\texttt{\textbf{cpyspec} rA, flags}
	%	\begin{itemize}
	%		\item Effect:  copy \texttt{flags} to \texttt{rA}.
	%	\end{itemize}
	%	\item Opcode \texttt{0x6}:
	%		\texttt{\textbf{cpyspec} ids, rA}
	%	\item Opcode \texttt{0x7}:
	%		\texttt{\textbf{cpyspec} rA, ids}
	%	\item Opcode \texttt{0x8}:
	%		\texttt{\textbf{cpyspec} ira, rA}
	%	\item Opcode \texttt{0x9}:
	%		\texttt{\textbf{cpyspec} rA, ira}
	%	\item Opcode \texttt{0xa}:
	%		\texttt{\textbf{cpyspec} ie, rA}
	%	\begin{itemize}
	%		\item Effect:  copy \texttt{rA[0]} to \texttt{ie}.
	%	\end{itemize}
	%	\item Opcode \texttt{0xb}:
	%		\texttt{\textbf{cpyspec} rA, ie}
	%	\begin{itemize}
	%		\item Effect:  copy \texttt{ie} to \texttt{rA}.
	%	\end{itemize}
	%	\item Opcode \texttt{0xc}:
	%		\texttt{\textbf{ei}}
	%	\begin{itemize}
	%		\item Effect:  copy \texttt{1} into \texttt{ie}.
	%	\end{itemize}
	%	\item Opcode \texttt{0xd}:
	%		\texttt{\textbf{di}}
	%	\begin{itemize}
	%		\item Effect:  copy \texttt{0} into \texttt{ie}.
	%	\end{itemize}
	%	\item Opcode \texttt{0xe}:
	%		\texttt{\textbf{push} ira}
	%	\begin{itemize}
	%		\item Effect:  pushes \texttt{ira} onto the stack,
	%		post-decrementing \texttt{sp}.
	%	\end{itemize}
	%	\item Opcode \texttt{0xf}:
	%		\texttt{\textbf{pop} ira}
	%	\begin{itemize}
	%		\item Effect:  pops \texttt{ira} off the stack,
	%		pre-incrementing \texttt{sp}.
	%	\end{itemize}
	%\end{itemize}
	%\doublespacing

	%\subsection{Instruction Group 7.2:  Multiplies, Divides,
	%Extra Interrupts Stuff, Push/Pop, and Index}
	%The following encoding is used, with each character representing one
	%bit:  \\
	%\texttt{1110 oooo aaaa 1000}, where

	%\singlespacing
	%\begin{itemize}
	%	\item \texttt{o} is the opcode
	%	\item \texttt{a} encodes register \texttt{rA} or one of registers 
	%	\texttt{hi}, \texttt{lo}, \texttt{flags}, denoted \texttt{sA},
	%	where \texttt{hi} is encoded as \texttt{0x0}, \texttt{lo} is
	%	encoded as \texttt{0x1}, and \texttt{flags} is encoded as
	%	\texttt{0x2}.
	%\end{itemize}
	%\doublespacing

	%Here is a list of instructions from this encoding group.

	%\singlespacing
	%\begin{itemize}
	%	\item Opcode \texttt{0x0}:
	%		\texttt{\textbf{mul} rA}
	%	\begin{itemize}
	%		\item Effect:  This instruction multiplies \texttt{lo} by
	%		\texttt{rA}, performing a 32-bit by 32-bit -> 32-bit multiply,
	%		storing result in \texttt{lo}.
	%	\end{itemize}
	%	\item Opcode \texttt{0x1}:
	%		\texttt{\textbf{udiv} rA}
	%	\begin{itemize}
	%		\item Effect:  This instruction performs a 32-bit by 32-bit ->
	%		32-bit unsigned divide of \texttt{lo} by \texttt{rA}, storing
	%		result in \texttt{lo}.
	%	\end{itemize}
	%	\item Opcode \texttt{0x2}:
	%		\texttt{\textbf{sdiv} rA}
	%	\begin{itemize}
	%		\item Effect:  This instruction performs a 32-bit by 32-bit ->
	%		32-bit signed divide of \texttt{lo} by \texttt{rA}, storing
	%		result in \texttt{lo}.
	%	\end{itemize}
	%	\item Opcode \texttt{0x3}:
	%		\texttt{\textbf{lumul} rA}
	%	\begin{itemize}
	%		\item Effect:  This instruction multiplies \texttt{lo} by
	%		\texttt{rA}, performing an unsigned 32-bit by 32-bit -> 64-bit
	%		multiply, storing result in \texttt{\{hi, lo\}}.
	%	\end{itemize}
	%	\item Opcode \texttt{0x4}:
	%		\texttt{\textbf{lsmul} rA}
	%	\begin{itemize}
	%		\item Effect:  This instruction multiplies \texttt{lo} by
	%		\texttt{rA}, performing a signed 32-bit by 32-bit -> 64-bit
	%		multiply, storing result in \texttt{\{hi, lo\}}.
	%	\end{itemize}
	%	\item Opcode \texttt{0x5}:
	%		\texttt{\textbf{reti}}
	%	\begin{itemize}
	%		\item Effect:  enables interrupts and performs
	%		\texttt{pc <= ira;}
	%	\end{itemize}
	%	\item Opcode \texttt{0x6}:
	%		\texttt{\textbf{jmp} ira}
	%	\begin{itemize}
	%		\item Effect:  \texttt{pc <= ira;}
	%	\end{itemize}
	%	\item Opcode \texttt{0x7}:
	%		\texttt{\textbf{push} rA}
	%	\begin{itemize}
	%		\item Effect:  pushes \texttt{rA} onto the stack,
	%		post-decrementing \texttt{sp}.
	%	\end{itemize}
	%	\item Opcode \texttt{0x8}:
	%		\texttt{\textbf{pop} rA}
	%	\begin{itemize}
	%		\item Effect:  pops \texttt{rA} off the stack,
	%		pre-incrementing \texttt{sp}.
	%	\end{itemize}
	%	\item Opcode \texttt{0x9}:
	%		\texttt{\textbf{push} sA}
	%	\begin{itemize}
	%		\item Effect:  pushes \texttt{sA} onto the stack,
	%		post-decrementing \texttt{sp}.
	%	\end{itemize}
	%	\item Opcode \texttt{0xa}:
	%		\texttt{\textbf{pop} sA}
	%	\begin{itemize}
	%		\item Effect:  pops \texttt{sA} off the stack,
	%		pre-incrementing \texttt{sp}.
	%	\end{itemize}
	%\end{itemize}
	%\doublespacing

	%\printbibliography[heading=bibnumbered,title={Bibliography}]

\end{document}

