\documentclass{article}

\usepackage{graphicx}
\usepackage{float}
\usepackage{fancyvrb}
\usepackage[T1]{fontenc}
\usepackage{lmodern}
\usepackage{setspace}
\usepackage[nottoc]{tocbibind}
\usepackage[font=large]{caption}
\usepackage{framed}
\usepackage{tabularx}
\usepackage{amsmath}
%\usepackage{hyperref}


\title{CPU Inspired By GSU}
\date{2018-10-31}
\author{FL4SHK}

%% Hide section, subsection, and subsubsection numbering
%\renewcommand{\thesection}{}
%\renewcommand{\thesubsection}{}
%\renewcommand{\thesubsubsection}{}


% Alternative form of doing section stuff
\renewcommand{\thesection}{}
\renewcommand{\thesubsection}{\arabic{section}.\arabic{subsection}}
\makeatletter
\def\@seccntformat#1{\csname #1ignore\expandafter\endcsname\csname the#1\endcsname\quad}
\let\sectionignore\@gobbletwo
\let\latex@numberline\numberline
\def\numberline#1{\if\relax#1\relax\else\latex@numberline{#1}\fi}
\makeatother

\makeatletter
\renewcommand\tableofcontents{%
    \@starttoc{toc}%
}
\makeatother

%Figures
%\begin{figure}[H]
%	\includegraphics[width=\linewidth]{example.png}
%\end{figure}

% Verbatim text
%\VerbatimInput{main.cpp}

%% Fixed-width text
%\texttt{module FullAdder(input logic a, b, c_in, output logic s, c_out);}
%% Bold
%\textbf{green eggs}
%% Italic
%\textit{and}
%% Underline
%\underline{eggs}

%% Non-numbered list
%\begin{itemize}
%\item item 0
%\item item 1
%\end{itemize}

%% Numbered list
%\begin{enumerate}
%\item item 0
%\item item 1
%\end{enumerate}

%% Spaces and new lines
%LaTeX ignores extra spaces and new lines.
%Place \\ at the end of a line to create a new line (but not create a new
%paragraph)

%% Use "\noindent" to prevent a paragraph from indenting

%% Tables
%\begin{table}[H]
%	\begin{center}
%		\caption{Results for \texttt{blocksPerGrid = 5}}:
%		\label{tab:table0}
%		\begin{tabular}{c|c}
%			\textbf{\texttt{threadsPerBlock}}
%				& \textbf{\texttt{scaling()} Running Time (us)}\\
%			\hline
%			32 & 156.39\\
%			64 & 163.59\\
%			128 & 155.62\\
%			256 & 155.56\\
%			512 & 161.57\\
%			1024 & 166.85\\
%		\end{tabular}
%	\end{center}
%\end{table}

\begin{document}
	\maketitle
	\pagenumbering{gobble}
	\newpage
	\pagenumbering{arabic}


	\doublespacing
	\section{Abstract}
	\setcounter{section}{-1}
	This is an instruction set designed to be similar to the SuperFX/GSU,
	while not being binary compatible, nor even assembly language
	compatible.  It takes some of the ideas from the GSU and runs with
	them.

	\newpage
	\singlespacing
	\section{Table of Contents}
	\tableofcontents
	\newpage

	\doublespacing
	\section{Introduction}
	\subsection{Registers}
	There are sixteen general-purpose registers:  \texttt{r0}, \texttt{r1},
	\texttt{r2}, \texttt{...}, \texttt{r12}, \texttt{lr}, \texttt{fp},
	\texttt{sp}.  From the hardware's perspective, they are all equivalent. 
	Each register is 32 bits long.  For special purpose registers, there
	are also \texttt{pc}, the program counter (which is 32 bits long), and
	the \texttt{flags}.  Also there are the interrupts-related registers:
	\texttt{ids} (the destination to go to upon an interrupt happening),
	\texttt{ira} (the program counter value to return to after an
	interrupt) and \texttt{ie} (whether or not interrupts are enabled).
	Two more registers are \texttt{hi} and \texttt{lo}, which are used as
	the high 32 bits and low 32 bits of the result of a 32 by 32 -> 64
	multiplication, or as the high 32 bits and low 32 bits of the result of
	a 64 by 32 -> 64 division.
	Here are the flags:

	\begin{table}[H]
		\begin{center}
			\caption{The Flags}
			\label{tab:flags}
			\begin{tabular}{|c|c|c|c|}
				\hline
				Zero (Z) & Carry (C) & oVerflow (V) & Negative (N)\\
				\hline
			\end{tabular}
		\end{center}
	\end{table}

	\newpage
	\section{Instruction Set}
	While this machine takes inspiration from the GSU, it uses 16-bit
	instructions rather than 8-bit ones.

	\subsection{Instruction Group 0:  The \texttt{pre} Instruction}
	The following encoding is used, with each character representing one
	bit:  \\
	\texttt{000i iiii iiii iiii}, where 
	
	\singlespacing
	\begin{itemize}
		\item \texttt{i} is a 13-bit constant.
	\end{itemize}

	At the assembly language level, the instruction is simply
	\texttt{pre expr}, but it isn't typical to use this instruction
	directly in assembly language source code.  Instead, when an immediate
	value will not fit in a 5-bit signed immediate, but will fit in an
	18-bit signed immediate, this instruction will be inserted by the
	assembler right before the instruction using an immediate value.  The
	13-bit constant will be used as bits \texttt{[17:5]} of the immediate
	value of the following instruction.  The full 18-bit immediate value
	will then be sign extended to 32 bits.  Also, if the previous
	instruction was a \texttt{pre} (or a \texttt{pick}), interrupts will
	not be serviced.

	\subsection{Instruction Group 1:  ALU Instr with Immediate}
	The following encoding is used, with each character representing one
	bit:  \\
	\texttt{001i iiii aaaa ooof}, where

	\singlespacing
	\begin{itemize}
		\item \texttt{i} is a 5-bit sign-extended immediate
		\item \texttt{a} encodes register \texttt{rA}
		\item \texttt{o} is the opcode
		\item \texttt{f} is \texttt{0} if this instruction cannot affect
		flags and \texttt{1} if this instruction is permitted to affect
		flags. 
	\end{itemize}
	\doublespacing

	Here is a list of instructions from this encoding group.

	\singlespacing
	\begin{itemize}
		\item Opcode \texttt{0x0}:
			\texttt{\textbf{addi} rA, simm}
		\begin{itemize}
			\item Mnemonic for when flags not affected:  \texttt{addi}
			\item Mnemonic for when flags affected:  \texttt{addi.f}
			\item Affectable flags:
				\texttt{Z}, \texttt{C}, \texttt{V}, \texttt{N}
		\end{itemize}
		\item Opcode \texttt{0x1}:
			\texttt{\textbf{subi} rA, simm}
		\begin{itemize}
			\item Mnemonic for when flags not affected:  \texttt{subi}
			\item Mnemonic for when flags affected:  \texttt{subi.f}
			\item Affectable flags:
				\texttt{Z}, \texttt{C}, \texttt{V}, \texttt{N}
		\end{itemize}
		\item Opcode \texttt{0x2}:
			\texttt{\textbf{addi} rA, pc, simm}
		\begin{itemize}
			\item Mnemonic for when flags not affected:  \texttt{addi}
			\item Mnemonic for when flags affected:  \texttt{addi.f}
			\item Affectable flags:
				\texttt{Z}, \texttt{C}, \texttt{V}, \texttt{N}
		\end{itemize}
		\item Opcode \texttt{0x3}:
			\texttt{\textbf{cpyi} rA, simm}
		\begin{itemize}
			\item Mnemonic for when flags not affected:  \texttt{cpyi}
			\item Mnemonic for when flags affected:  \texttt{cpyi.f}
			\item Note:  Copy an immediate value into \texttt{rA}
			\item Affectable flags:
				\texttt{Z}, \texttt{N}
		\end{itemize}
		\item Opcode \texttt{0x4}:
			\texttt{\textbf{lui} rA, simm}
		\begin{itemize}
			\item Mnemonic for when flags not affected:  \texttt{lui}
			\item Mnemonic for when flags affected:  \texttt{lui.f}
			\item Note:  Copy bits \texttt{[15:0]} of the sign-extended
			immediate value into bits \texttt{[31:16]} of \texttt{rA}.
			Bits \texttt{[15:0]} of \texttt{rA} are not affected.
			\item Affectable flags:
				\texttt{Z}, \texttt{N}
		\end{itemize}
		\item Opcode \texttt{0x5}:
			\texttt{\textbf{lsli} rA, simm}
		\begin{itemize}
			\item Mnemonic for when flags not affected:  \texttt{lsli}
			\item Mnemonic for when flags affected:  \texttt{lsli.f}
			\item Note:  Logical shift left
			\item Affectable flags:
				\texttt{Z}, \texttt{N}
		\end{itemize}
		\item Opcode \texttt{0x6}:
			\texttt{\textbf{lsri} rA, simm}
		\begin{itemize}
			\item Mnemonic for when flags not affected:  \texttt{lsri}
			\item Mnemonic for when flags affected:  \texttt{lsri.f}
			\item Note:  Logical shift right
			\item Affectable flags:
				\texttt{Z}, \texttt{N}
		\end{itemize}
		\item Opcode \texttt{0x7}:
			\texttt{\textbf{asri} rA, simm}
		\begin{itemize}
			\item Mnemonic for when flags not affected:  \texttt{asri}
			\item Mnemonic for when flags affected:  \texttt{asri.f}
			\item Note:  Arithmetic shift right
			\item Affectable flags:
				\texttt{Z}, \texttt{N}
		\end{itemize}
	\end{itemize}

	\doublespacing

	\subsection{Instruction Group 2:  ALU Instr without Immediate}
	The following encoding is used, with each character representing one
	bit:  \\
	\texttt{010o cccc aaaa ooof}, where

	\singlespacing
	\begin{itemize}
		\item \texttt{o} is the opcode
		\item \texttt{c} encodes register \texttt{rC}
		\item \texttt{a} encodes register \texttt{rA}
		\item \texttt{f} is \texttt{0} if this instruction cannot affect
		flags and \texttt{1} if this instruction is permitted to affect
		flags. 
	\end{itemize}
	\doublespacing

	Here is a list of instructions from this encoding group.

	\singlespacing
	\begin{itemize}
		\item Opcode \texttt{0x0}:
			\texttt{\textbf{add} rA, rC}
		\begin{itemize}
			\item Mnemonic for when flags not affected:  \texttt{add}
			\item Mnemonic for when flags affected:  \texttt{add.f}
			\item Affectable flags:
				\texttt{Z}, \texttt{C}, \texttt{V}, \texttt{N}
		\end{itemize}
		\item Opcode \texttt{0x1}:
			\texttt{\textbf{sub} rA, rC}
		\begin{itemize}
			\item Mnemonic for when flags not affected:  \texttt{sub}
			\item Mnemonic for when flags affected:  \texttt{sub.f}
			\item Affectable flags:
				\texttt{Z}, \texttt{C}, \texttt{V}, \texttt{N}
		\end{itemize}
		\item Opcode \texttt{0x2}:
			\texttt{\textbf{add} rA, pc, rC}
		\begin{itemize}
			\item Mnemonic for when flags not affected:  \texttt{add}
			\item Mnemonic for when flags affected:  \texttt{add.f}
			\item Affectable flags:
				\texttt{Z}, \texttt{C}, \texttt{V}, \texttt{N}
		\end{itemize}
		\item Opcode \texttt{0x3}:
			\texttt{\textbf{cpy} rA, rC}
		\begin{itemize}
			\item Mnemonic for when flags not affected:  \texttt{cpy}
			\item Mnemonic for when flags affected:  \texttt{cpy.f}
			\item Note:  Copy \texttt{rC} into \texttt{rA}
			\item Affectable flags:
				\texttt{Z}, \texttt{N}
		\end{itemize}
		\item Opcode \texttt{0x4}:
			\texttt{\textbf{lur} rA, rC}
		\begin{itemize}
			\item Mnemonic for when flags not affected:  \texttt{lur}
			\item Mnemonic for when flags affected:  \texttt{lur.f}
			\item Note:  Copy \texttt{b[15:0]} to \texttt{a[31:16]}.
			Bits \texttt{[15:0]} of \texttt{rA} are not affected.
			\item Affectable flags:
				\texttt{Z}, \texttt{N}
		\end{itemize}
		\item Opcode \texttt{0x5}:
			\texttt{\textbf{lsl} rA, rC}
		\begin{itemize}
			\item Mnemonic for when flags not affected:  \texttt{lsl}
			\item Mnemonic for when flags affected:  \texttt{lsl.f}
			\item Note:  Logical shift left
			\item Affectable flags:
				\texttt{Z}, \texttt{N}
		\end{itemize}
		\item Opcode \texttt{0x6}:
			\texttt{\textbf{lsr} rA, rC}
		\begin{itemize}
			\item Mnemonic for when flags not affected:  \texttt{lsr}
			\item Mnemonic for when flags affected:  \texttt{lsr.f}
			\item Note:  Logical shift right
			\item Affectable flags:
				\texttt{Z}, \texttt{N}
		\end{itemize}
		\item Opcode \texttt{0x7}:
			\texttt{\textbf{asr} rA, rC}
		\begin{itemize}
			\item Mnemonic for when flags not affected:  \texttt{asr}
			\item Mnemonic for when flags affected:  \texttt{asr.f}
			\item Note:  Arithmetic shift right
			\item Affectable flags:
				\texttt{Z}, \texttt{N}
		\end{itemize}
		\item Opcode \texttt{0x8}:
			\texttt{\textbf{adc} rA, rC}
		\begin{itemize}
			\item Mnemonic for when flags not affected:  \texttt{adc}
			\item Mnemonic for when flags affected:  \texttt{adc.f}
			\item Note:  Add with carry
			\item Affectable flags:
				\texttt{Z}, \texttt{C}, \texttt{V}, \texttt{N}
		\end{itemize}
		\item Opcode \texttt{0x9}:
			\texttt{\textbf{sbc} rA, rC}
		\begin{itemize}
			\item Mnemonic for when flags not affected:  \texttt{sbc}
			\item Mnemonic for when flags affected:  \texttt{sbc.f}
			\item Note:  Subtract with Borrow
			\item Affectable flags:
				\texttt{Z}, \texttt{C}, \texttt{V}, \texttt{N}
		\end{itemize}
		\item Opcode \texttt{0xa}:
			\texttt{\textbf{cmp} rA, rC}
		\begin{itemize}
			\item Note:  Compare \texttt{rA} to \texttt{rC}.  \texttt{cmp}
			\textit{always} is able to affect flags, independent of the
			encoded \texttt{f} bit of the instruction.
			\item Affectable flags:
				\texttt{Z}, \texttt{C}, \texttt{V}, \texttt{N}
		\end{itemize}
		\item Opcode \texttt{0xb}:
			\texttt{\textbf{and} rA, rC}
		\begin{itemize}
			\item Mnemonic for when flags not affected:  \texttt{and}
			\item Mnemonic for when flags affected:  \texttt{and.f}
			\item Note:  Bitwise AND
			\item Affectable flags:
				\texttt{Z}, \texttt{N}
		\end{itemize}
		\item Opcode \texttt{0xc}:
			\texttt{\textbf{orr} rA, rC}
		\begin{itemize}
			\item Mnemonic for when flags not affected:  \texttt{orr}
			\item Mnemonic for when flags affected:  \texttt{orr.f}
			\item Note:  Bitwise OR
			\item Affectable flags:
				\texttt{Z}, \texttt{N}
		\end{itemize}
		\item Opcode \texttt{0xd}:
			\texttt{\textbf{xor} rA, rC}
		\begin{itemize}
			\item Mnemonic for when flags not affected:  \texttt{xor}
			\item Mnemonic for when flags affected:  \texttt{xor.f}
			\item Note:  Bitwise XOR
			\item Affectable flags:
				\texttt{Z}, \texttt{N}
		\end{itemize}
		%\item Opcode \texttt{0xe}:
		%	\texttt{\textbf{mul} rA, rC}
		%\begin{itemize}
		%	\item Mnemonic for when flags not affected:  \texttt{mul}
		%	\item Mnemonic for when flags affected:  \texttt{mul.f}
		%	\item Affectable flags:
		%		\texttt{Z}, \texttt{N}
		%\end{itemize}
		\item Opcode \texttt{0xe}:
			\texttt{\textbf{ludiv} rA}
		\begin{itemize}
			\item Mnemonic for when flags not affected:  \texttt{ludiv}
			\item Mnemonic for when flags affected:  \texttt{ludiv.f}
			\item Note:  performs a 64-bit by 32-bit unsigned division of
			\texttt{\{hi, lo\}} by \texttt{rA}, storing result in
			\texttt{\{hi, lo\}}.
			\item Affectable flags:
				\texttt{Z}, \texttt{N}
		\end{itemize}
		\item Opcode \texttt{0xf}:
			\texttt{\textbf{lsdiv} rA}
		\begin{itemize}
			\item Mnemonic for when flags not affected:  \texttt{lsdiv}
			\item Mnemonic for when flags affected:  \texttt{lsdiv.f}
			\item Note:  performs a 64-bit by 32-bit signed division of
			\texttt{\{hi, lo\}} by \texttt{rA}, storing result in
			\texttt{\{hi, lo\}}.
			\item Affectable flags:
				\texttt{Z}, \texttt{N}
		\end{itemize}
	\end{itemize}

	\doublespacing
	\subsection{Instruction Group 3:  Pick \texttt{rB}}
	The following encoding is used, with each character representing one
	bit:  \\
	\texttt{0110 0000 0000 bbbb}, where

	\singlespacing
	\begin{itemize}
		\item \texttt{b} encodes register \texttt{rB}
	\end{itemize}

	At the assembly language level, this instruction is simply
	\texttt{pick rB}, but as with \texttt{pre}, it isn't typical to use
	this instruction directly in assembly language source code.  Instead,
	when an instruction is typed as \texttt{instr rA, rB, operand}
	\textit{and} when \texttt{rA} is not the same register as \texttt{rB},
	a \texttt{pick rB} instruction will be inserted by the assembler before
	the instruction in the source code.  As mentioned before, when the
	instruction previous to the current one is either a \texttt{pre} or a
	\texttt{pick}, interrupts will not be serviced.  Note that since
	\texttt{pre}'s 13-bit partial immediate only affects the instruction
	following the \texttt{pre}, only a 5-bit signed immediate can be used
	with \texttt{pick}.

	For all instructions, the default value of \texttt{rB} is \texttt{rA},
	except when the previous instruction was a \texttt{pick} instruction.

	%\subsection{Instruction Group 4:  Relative Branches}
	%The following encoding is used, with each character representing one
	%bit:  \\
	%\texttt{0111 oooo iiii iiii}, where


\end{document}

