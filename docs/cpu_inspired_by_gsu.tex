\documentclass{article}

\usepackage{graphicx}
\usepackage{float}
\usepackage{fancyvrb}
\usepackage[T1]{fontenc}
\usepackage{lmodern}
%\usepackage{setspace}
%\usepackage[nottoc]{tocbibind}
\usepackage[font=large]{caption}
%\usepackage{framed}
%\usepackage{tabularx}
%\usepackage{amsmath}
%\usepackage{hyperref}


\title{CPU Inspired By GSU}
\date{2018-10-31}
\author{FL4SHK}

%% Hide section, subsection, and subsubsection numbering
%\renewcommand{\thesection}{}
%\renewcommand{\thesubsection}{}
%\renewcommand{\thesubsubsection}{}


% Alternative form of doing section stuff
\renewcommand{\thesection}{}
\renewcommand{\thesubsection}{\arabic{section}.\arabic{subsection}}
\makeatletter
\def\@seccntformat#1{\csname #1ignore\expandafter\endcsname\csname the#1\endcsname\quad}
\let\sectionignore\@gobbletwo
\let\latex@numberline\numberline
\def\numberline#1{\if\relax#1\relax\else\latex@numberline{#1}\fi}
\makeatother

\makeatletter
\renewcommand\tableofcontents{%
    \@starttoc{toc}%
}
\makeatother

%Figures
%\begin{figure}[H]
%	\includegraphics[width=\linewidth]{example.png}
%\end{figure}

% Verbatim text
%\VerbatimInput{main.cpp}

%% Fixed-width text
%\texttt{module FullAdder(input logic a, b, c_in, output logic s, c_out);}
%% Bold
%\textbf{green eggs}
%% Italic
%\textit{and}
%% Underline
%\underline{eggs}

%% Non-numbered list
%\begin{itemize}
%\item item 0
%\item item 1
%\end{itemize}

%% Numbered list
%\begin{enumerate}
%\item item 0
%\item item 1
%\end{enumerate}

%% Spaces and new lines
%LaTeX ignores extra spaces and new lines.
%Place \\ at the end of a line to create a new line (but not create a new
%paragraph)

%% Use "\noindent" to prevent a paragraph from indenting

%% Tables
%\begin{table}[H]
%	\begin{center}
%		\caption{Results for \texttt{blocksPerGrid = 5}}:
%		\label{tab:table0}
%		\begin{tabular}{c|c}
%			\textbf{\texttt{threadsPerBlock}}
%				& \textbf{\texttt{scaling()} Running Time (us)}\\
%			\hline
%			32 & 156.39\\
%			64 & 163.59\\
%			128 & 155.62\\
%			256 & 155.56\\
%			512 & 161.57\\
%			1024 & 166.85\\
%		\end{tabular}
%	\end{center}
%\end{table}

\begin{document}
	\maketitle
	\pagenumbering{gobble}
	\newpage
	\pagenumbering{arabic}


	\section{Introduction}
	This is an instruction set designed to be similar to the SuperFX/GSU,
	while not being binary compatible, nor even assembly language
	compatible.  It takes some of the ideas from the GSU and runs with
	them.

	\newpage
	\tableofcontents
	\newpage

	\subsection{Registers}
	There are sixteen general-purpose registers:  \texttt{r0}, \texttt{r1},
	\texttt{r2}, \texttt{...}, \texttt{r12}, \texttt{lr}, \texttt{fp},
	\texttt{sp}.  From the hardware's perspective, they are all equivalent. 
	Each register is 32 bits long.  For special purpose registers, there
	are also \texttt{pc}, the program counter (which is 32 bits long), and
	the \texttt{flags}.  Also there are the interrupts-related registers:
	\texttt{ids} (the destination to go to upon an interrupt happening),
	\texttt{ira} (the program counter value to return to after an
	interrupt) and \texttt{ie} (whether or not interrupts are enabled).
	Here are the flags:

	\begin{table}[H]
		\begin{center}
			\caption{The Flags}
			\label{tab:flags}
			\begin{tabular}{|c|c|c|c|}
				\hline
				Zero (Z) & Carry (C) & oVerflow (V) & Negative (N)\\
				\hline
			\end{tabular}
		\end{center}
	\end{table}

\end{document}

